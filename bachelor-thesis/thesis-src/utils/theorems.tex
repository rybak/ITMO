\theoremstyle{plain}
\newtheorem{theorem}{Теорема}
\newtheorem{prop}[theorem]{Утверждение}
\newtheorem{corollary}[theorem]{Следствие}
\newtheorem{lemma}[theorem]{Лемма}
\newtheorem{question}[theorem]{Вопрос}
\newtheorem{conjecture}[theorem]{Гипотеза}
\newtheorem{assumption}[theorem]{Предположение}

\theoremstyle{definition}
\newtheorem{definition}[theorem]{Определение}
\newtheorem{notation}[theorem]{Обозначение}
\newtheorem{condition}[theorem]{Условие}
\newtheorem{example}[theorem]{Пример}
%\newtheorem{algorithm}[theorem]{Алгоритм}
\floatname{algorithm}{Листинг}
\renewcommand{\algorithmicrequire}{\textbf{Вход:}}

%\newtheorem{introduction}[theorem]{Introduction}

\renewcommand{\proof}{\\\textbf{Доказательство.}~}
% \renewcommand{\lstlistingname}{Листинг}
 
% \lstnewenvironment{snippet}[1][]%
% {
%    \noindent
%    \minipage{\linewidth} 
%    \vspace{0.5\baselineskip}
%    \lstset{basicstyle=\ttfamily\footnotesize,frame=single,#1}}
% {\endminipage}

%%%%%%%%%%%%%%%%%%%%%%%%%%%%%%%%%%%%%%%%%%%%%%%%%%%%%%%%%%%%%%%%%%%%%%%%%%%%%%%

\numberwithin{theorem}{chapter}        % Numbers theorems "x.y" where x
                                        % is the section number, y is the
                                        % theorem number

%\renewcommand{\thetheorem}{\arabic{chapter}.\arabic{theorem}}

%\makeatletter                          % This sequence of commands will
%\let\c@equation\c@theorem              % incorporate equation numbering
%\makeatother                           % into the theorem numbering scheme

%\renewcommand{\theenumi}{(\roman{enumi})}

%%%%%%%%%%%%%%%%%%%%%%%%%%%%%%%%%%%%%%%%%%%%%%%%%%%%%%%%%%%%%%%%%%%%%%%%%%%%%%


