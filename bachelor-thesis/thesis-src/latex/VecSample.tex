\begin{code}\>\<%
\\
\>\AgdaKeyword{module} \AgdaModule{VecSample} \AgdaKeyword{where}\<%
\\
\>\AgdaKeyword{data} \AgdaDatatype{ℕ} \AgdaSymbol{:} \AgdaPrimitiveType{Set} \AgdaKeyword{where}\<%
\\
\>[0]\AgdaIndent{2}{}\<[2]%
\>[2]\AgdaInductiveConstructor{zero} \AgdaSymbol{:} \AgdaDatatype{ℕ}\<%
\\
\>[0]\AgdaIndent{2}{}\<[2]%
\>[2]\AgdaInductiveConstructor{succ} \AgdaSymbol{:} \AgdaDatatype{ℕ} \AgdaSymbol{→} \AgdaDatatype{ℕ}\<%
\\
\>\AgdaKeyword{data} \AgdaDatatype{Vec} \AgdaBound{A} \AgdaSymbol{:} \AgdaDatatype{ℕ} \AgdaSymbol{→} \AgdaPrimitiveType{Set} \AgdaKeyword{where}\<%
\\
\>[0]\AgdaIndent{2}{}\<[2]%
\>[2]\AgdaInductiveConstructor{nil} \<[7]%
\>[7]\AgdaSymbol{:} \AgdaDatatype{Vec} \AgdaBound{A} \AgdaInductiveConstructor{zero}\<%
\\
\>[0]\AgdaIndent{2}{}\<[2]%
\>[2]\AgdaInductiveConstructor{cons} \AgdaSymbol{:} \AgdaSymbol{∀} \AgdaSymbol{\{}\AgdaBound{n}\AgdaSymbol{\}} \AgdaSymbol{→} \AgdaBound{A} \AgdaSymbol{→} \AgdaDatatype{Vec} \AgdaBound{A} \AgdaBound{n} \AgdaSymbol{→} \AgdaDatatype{Vec} \AgdaBound{A} \AgdaSymbol{(}\AgdaInductiveConstructor{succ} \AgdaBound{n}\AgdaSymbol{)}\<%
\\
\>\<\end{code}

Такое определение позволяет нам описать функцию $ \F{head} $ для такого списка, которая не может бросить исключение:
\begin{code}\>\<%
\\
\>\AgdaFunction{head} \AgdaSymbol{:} \AgdaSymbol{∀} \AgdaSymbol{\{}\AgdaBound{A}\AgdaSymbol{\}} \AgdaSymbol{\{}\AgdaBound{n}\AgdaSymbol{\}} \AgdaSymbol{→} \AgdaDatatype{Vec} \AgdaBound{A} \AgdaSymbol{(}\AgdaInductiveConstructor{succ} \AgdaBound{n}\AgdaSymbol{)} \AgdaSymbol{→} \AgdaBound{A}\<%
\\
\>\<\end{code}
У аргумента функции $ \F{head} $ тип $ \D{Vec}\,A\,(\DC{succ}\,n) $, то есть вектор, в котором есть хотя бы один элемент.
Это позволяет произвести сопоставление с образцом только по конструктору $ \DC{cons} $:
\begin{code}\>\<%
\\
\>\AgdaFunction{head} \AgdaSymbol{(}\AgdaInductiveConstructor{cons} \AgdaBound{a} \AgdaBound{as}\AgdaSymbol{)} \AgdaSymbol{=} \AgdaBound{a}\<%
\\
\>\<\end{code}
