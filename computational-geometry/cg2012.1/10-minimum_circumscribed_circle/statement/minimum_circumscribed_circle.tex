\begin{problem}{10. Вычисление минимальной описанной окружности множества точек плоскости}{standard input}{standard output}{2 секунды}{256 мегабайт}

На плоскости задано $n$ точек. Требуется найти окружность минимального радиуса, содержащую все точки.

\InputFile

В первой строчке входного файла записано число $n$ --- количество точек. В последующих $n$ строчках через пробел записаны пары вещественных чисел $(x_i, y_i)$, задающих координаты точек на плоскости.

\OutputFile

В выходной файл выведите три числа $x, y, R$ --- координаты центра и радиус минимальной окружности соответственно.

\Examples

\begin{example}%
\exmp{
5
2 2
-0.5 1
1 3
0.8 -0.2
3 1
}{
1 1 2
}%
\end{example}

\end{problem}
