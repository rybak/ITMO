\begin{problem}{14. Триангуляция многоугольника}{standard input}{standard output}{2 секунды}{256 мегабайт}

На плоскости задан произвольный многоугольник с дырками. Триангуляцией многоугольника $P$ назовём его разбиение на множество попарно непересекающихся треугольников, объединение которых даёт  $P$. В данной задаче требуется найти триангуляцию многоугольника.

\InputFile

На первой строчке входного файла записано натуральное число $n$ -- количество точек внешнего контура многоугольника. На последующих  $n$ строчках через пробел записаны пары вешественных чисел $(x_i, y_i)$, задающих координаты точек на плоскости. Вершины обходятся против часовой стрелки. На следующей строчке записано натуральное число $m$ -- количество дырок. Каждая дырка задаётся в формате: натуральное число $k$ -- количество точек в дырке и  на последующих  $k$ строчках через пробел записаны пары вешественных чисел $(x_i, y_i)$, задающих координаты точек на плоскости.  Вершины дырок обходятся по часовой стрелке.
\OutputFile

В первой строчке выведите натуральное число  $p$ --- число треугольников в триангуляции. На следующих $p$ строчках выведите номера вершин против часовой стрелки для каждого треугольника. Вершины нумеруются с единицы в порядке появления в исходных данных. Если триангуляций несколько, выведите любую из них.

\Examples

\begin{example}%
\exmp{
4
0 0
2 0
2 2
0 2
0
}{
2
1 2 3
1 3 4
}%
\exmp{
4
0 0
2 0
2 2
0 2
1
4
0.5 0.5
0.5 1.5
1.5 1.5
1.5 0.5
}{
8
4 5 6
4 1 5
1 2 5
2 5 8
2 3 8
3 7 8
3 4 7
4 6 7
}%
\end{example}

\end{problem}