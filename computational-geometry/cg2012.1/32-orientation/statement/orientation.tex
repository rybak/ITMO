\begin{problem}{31. Предикат «поворот»}{standard input}{standard output}{2 секунды}{256 мегабайт}

Предикат «поворот» — геометрический предикат, принимающий три точки $a$, $b$ и $c$ на плоскости и определяющийся следующим образом:\\
$\operatorname{turn}(a, b, c) = 
\begin{cases}
$1$ ,& \text{если $c$ лежит слева от прямой, заданной точками $a$ и $b$} \\
$0$ ,& \text{если $c$ лежит на прямой, заданной точками $a$ и $b$} \\
$-1$,& \text{если $c$ лежит справа от прямой, заданной точками $a$ и $b$} 
\end{cases}
$

\InputFile

В первой строке входного файла задано натуральное число $n$ — количество тестов. \\
В следующих $3n$ строках входного файла заданы пары вещественных чисел — координаты точек $a$, $b$ и $c$. Гарантируется, что точки $a$ и $b$ не совпадают. \\
Примечание: во всех тестах на корректность $n = 1$. Мультитест сделан для удобства тестирования на производительность.


\OutputFile
В выходной файл через пробел выведите $n$ чисел — значений предиката для заданых точек.

\Examples

\begin{example}%
\exmp{
1
2 1.5
0.5 0.5
3.5 2.5
}{
0
}%
\end{example}

\begin{example}%
\exmp{
1
2 1.5
0.5 0.5
1.234 0.1123
}{
1
}%
\end{example}

\begin{example}%
\exmp{
1
2 1.5
0.5 0.5
5.21341 324124.5
}{
-1
}%
\end{example}

\end{problem}
