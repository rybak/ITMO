\begin{problem}{15. Пересечение множества отрезков}{standard input}{standard output}{2 секунды}{256 мегабайт}

На плоскости задано множество отрезков. Требуется найти все точки пересечения.
При наложении отрезков считать две точки, концы общей части этих отрекзов.
Каждую точку нужно задать отрезками проходящими через нее.
\InputFile
В первой строке задано целое число $n$ --- количество отрезков.
Следующие $n$ строк содержат по 4 вещественных числа --- координаты начала и конца $i$-го отрезка.

\OutputFile
В первую строку выведите число $m$ --- количество точек пересечения отрезков.
В следующие $m$ строчек выведите количество отрезков образующих $i$-ую точку пересечения и номера этих отреков. Отрезки нумеруются с единицы в порядке появления в исходных данных.  
\Examples

\begin{example}%
\exmp{
4
4.0 4.0 11.0 11.0
1.0 9.0 15.0 9.0
6.0 12.0 14.0 4.0
2.0 12.0 2.0 3.0
}{
2
3 1 2 3
2 2 4
}%
\end{example}

\end{problem}