\begin{problem}{01. Выпуклая оболочка точек плоскости}{standard input}{standard output}{2 секунды}{256 мегабайт}

Задано множество точек плоскости. Требуется найти минимальный по включению выпуклый многоугольник, содержащий все точки из заданного множества.

\InputFile

На первой строчке входного файла записано число $n$ — количество точек в множестве. На последующих $n$ строчках записаны пары вещественных чисел $(x_i, y_i)$, задающие координаты элементов множества на плоскости.

\OutputFile

На первой строке выходного файла выведите $m$ — количество вершин многоугольника. Далее, на $m$ строчках выведите координаты точек, являющихся вершинами многоугольника в порядке обхода против часовой стрелки. 

\Examples

\begin{example}%
\exmp{
5
0.0 0.0
0.0 1.0
1.0 0.0
1.0 1.0
0.5 0.5
}{
4
0.0 0.0
1.0 0.0
1.0 1.0
0.0 1.0
}%
\end{example}

\end{problem}
