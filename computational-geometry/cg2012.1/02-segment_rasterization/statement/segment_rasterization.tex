\begin{problem}{02. Растеризация отрезка}{standard input}{standard output}{2 секунды}{256 мегабайт}

Пусть на $\mathbb R^2$ задана координатная сетка: плоскость разбита на непересекающиеся ячейки вида $[n; n + 1) \times [m; m + 1)$, где $n$ и $m$ --- целые числа. Растеризацией объекта назовём последовательность ячеек, пересекающих отрезок, упорядоченную лексикографически по координатам $(x, y)$ нижнего левого угла.

\InputFile

В первых двух строчках входных данных заданы концы отрезка своими координатами.

\OutputFile

В первой строчке выходных данных требуется вывести количество ячеек плоскости, которые пересекает исходный отрезок. Далее требуется вывести координаты нижнего левого угла для каждой ячейки плоскости, входящей в растеризацию отрезка, на новой строчке для каждой ячейки.

\Examples

\begin{example}%
\exmp{
0.5 0.5
2.5 2.5
}{
3
0 0
1 1
2 2
}%
\exmp{
0.5 0.5
2.5 -1.5
}{
5
0 0
1 0
1 -1
2 -1
2 -2
}%
\end{example}

\end{problem}
