\begin{problem}{13. Триангуляция Делоне множества точек}{standard input}{standard output}{2 секунды}{256 мегабайт}

На плоскости задано множество точек. Триангуляцией называется планарный граф, все внутренние области которого являются треугольниками. Триангуляцией Делоне множества точек $S$ называют триангуляцию $DT(S)$, в которой:
\begin{itemize}
\item никакая точка $a$ из $S$ не содержится в окружности, описанной вокруг любого треугольника $T$ из $DT(S)$
\item точка $a$ не является ни одной из вершин $T$ .
\end{itemize}

\InputFile

На первой строчке входного файла записано натуральное число $n$ --- количество точек в множестве. На последующих $n$ строчках через пробел записаны пары вещественных чисел $(x_i, y_i)$, задающих координаты точек на плоскости.

\OutputFile

В первой строчке выведите натуральное число $p$ --- число треугольников в триангуляции. На следующих $p$ строчках выведите номера точек против часовой стрелки для каждого треугольника.
Вершины нумеруются с единицы в порядке появления в исходных данных. Если триангуляций несколько, выведите любую из них.

\Examples

\begin{example}%
\exmp{
4
0 0
2 0
2 2
0 2
}{
2
1 2 3
1 3 4
}%
\exmp{
8
0 0
2 0
2 2
0 2
0.5 0.5
0.5 1.5
1.5 1.5
1.5 0.5
}{
8
4 5 6
4 1 5
1 2 5
2 5 8
2 3 8
3 7 8
3 4 7
4 6 7
}%
\end{example}

\end{problem}