\begin{problem}{Рыцарский щит}{shield.in}{shield.out}{2 секунды}

%Автор разбора: Алексей Цыпленков

\renewcommand{\min}{min}

\begin{flushright}
\emph{Автор задачи: Владимир Ульянцев}

\emph{Автор разбора: Алексей Цыпленков}
\end{flushright}

Пусть $S_1$ и $S_2$ - периметры первого и второго треугольников соответственно.
Тогда при совмещении сторон с длинами $a$ и $b$ периметр получившейся фигуры 
будет равен $P = (S_1 - a) + (S_2 - b) + |a - b| = S_1 + S_2 - a - b + |a - b| =
S_1 + S_2 - 2 * min(a, b)$.
Очевидно, что значение $P$ минимально, когда значение $min(a, b)$ максимально,
что можно достичь, совместив наидлиннейшие стороны первого и второго треугольников.

Приведем пример решения на языке Паскаль:

\begin{lstlisting}
    for i := 1 to 3 do
        read(first[i]);
    for i := 1 to 3 do
        read(second[i]);

    ans := 0;
    for i := 1 to 3 do
    begin
        ans := ans + first[i];
        ans := ans + second[i];
    end;

    max1 := -1;
    max2 := -1;

    for i := 1 to 3 do
    begin
        max1 := max(max1, first[i]);
        max2 := max(max2, second[i]);
    end;
    writeln(ans - 2 * min(max1, max2) );
\end{lstlisting}
\end{problem}
