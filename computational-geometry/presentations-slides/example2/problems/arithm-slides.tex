\begin{frame}
  \frametitle{Задача <<Арифметическая прогрессия>>}
  \begin{center}
    { \Huge $17, 239, 461, \ldots$ }
  \end{center}
\end{frame}

\begin{frame}
  \frametitle{Над задачей работали}
  \begin{itemize}
    \item Идея задачи: Андрей Станкевич
    \item Текст условия: Сергей Мельников
    \item Тесты, проверяющая программа и др.: Антон Ахи
    \item Решения: Владимир Ульянцев, Сергей Мельников
    \item Текст разбора: Сергей Мельников
  \end{itemize}
\end{frame}

\begin{frame}
  \frametitle{Формулировка задачи}
  \begin{itemize}
    \item Дано $2n$ чисел, из них надо выбрать $n$ чисел, образующих арифметическую прогрессию.
    \item Гарантируется, что это можно сделать.
  \end{itemize}
\end{frame}

\begin{frame}
  \frametitle{Идея решения}
  \begin{itemize}
    \item Если выбрать случайный элемент исходной последовательности,
    	 то с вероятностью $\frac{1}{2}$ он будет принадлежать арифметической прогрессии.
    \item Если выбрать два случайных элемента $a$ и $b$, то оба будут принадлежать арифметической прогрессии с вероятностью $\frac{1}{4}$.
    \item Разность арифметической будет делителем $a - b$.
    \item Переберем все делители числа $a - b$ за $O(\sqrt{n})$.
  \end{itemize}
\end{frame}

\begin{frame}
  \frametitle{Идея решения}
  \begin{itemize}
	\item Знаем элемент $a$ и разность $d$, проверим какая максимальная длина арифметической прогрессии может быть.
	\item Для этого отсортируем исходный массив.
	\item Можно проверять, принадлежит ли элемент исходной последовательности, двоичным поиском.
	\item Будем уменьшать число $a$ на $d$, пока оно будет принадлежать исходной последовательности.
	\item Будем увеличивать $a$ на $d$, пока оно будет принадлежать исходной последовательности.
	\item Если есть не менее $n$ элементов, то это ответ.
  \end{itemize}
\end{frame}



\begin{frame}
  \frametitle{Итого}
  \begin{itemize}
    % \item % Сюда можно вставить статистику, если будет время после контеста
    \item Выбираем случайные элементы, с вероятностью $\frac{1}{4}$ мы выбираем элементы из арифметической прогрессии.
    \item Перебор всех делителей $O(\sqrt{n})$
    \item У чисел до $10^9$ не более нескольких сотен делителей.
    \item Проверка ответа $O(n \log n)$
    \item Вопросы?
  \end{itemize}
\end{frame}

