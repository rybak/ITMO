\begin{problem}{06. Растеризация треугольника}{standard input}{standard output}{2 секунды}{256 мегабайт}

Пусть на $\mathbb R^2$ задана координатная сетка: плоскость разбита на непересекающиеся ячейки вида $[n; n + 1) \times [m; m + 1)$, где $n$ и $m$ --- целые числа. Растеризацией объекта назовём множество ячеек, перечисленных в лексикографическом порядке, которые лежат внутри или на границе объекта. В данной задаче требуется найти растеризацию треугольника.

\InputFile

В первых трех строчках входных данных заданы координаты вершин треугольника.

\OutputFile

В первой строчке выходных данных требуется вывести количество ячеек плоскости, которые участвуют в растеризации. Далее требуется на каждой строчке вывести ячейки плоскости (ячейка задаётся координатами нижнего левого угла) в лексикографическом порядке. Если в перечислении при фиксированной координате $x$ следует несколько ячеек подряд, необходимо вывести только ячейки с минимальной и максимальной координатами $y$.

\Examples

\begin{example}%
\exmp{
0.5 0.5
4 0
3 3
}{
9
0 0
1 0
1 1
2 0
2 2
3 0
3 2
}%
\end{example}

\end{problem}
