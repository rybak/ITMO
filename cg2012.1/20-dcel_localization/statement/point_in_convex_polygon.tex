\begin{problem}{05. Массовая принадлежность точки выпуклому прямоугольнику}{standard input}{standard output}{2 секунды}{256 мегабайт}

На плоскости задан планарный граф. Требуется на каждый запрос локализации выдать ячейку минимальной размерности, содержащую точку  запроса.  

\InputFile

В первой строчке входного файла записано три натуральных числа $n_p$, $n_s$ и $n_f$ --- количество вершин, отрезков и граней графа соответственно. На последующих $n_s$ строчках заданы вершины.  Вершины задаются парами вещественных чисел $(x_i, y_i)$, задающих координаты точек.   На следующих $n_s$ строчках заданы отрезки. Каждый отрезкок задается номерами двух вершин. На следующих $n_f$ строчках задаются грани. Каждая грань задается количестом отрезков образующих её и номерами этих отрезков. На следующей строчке записано натуральное число $m$ --- количество запросов. На последующих $m$ строчках следуют пары вещественных чисел $(x_i, y_i)$, задающих координаты точек-запросов. Объекты нумерются с 0 в порядке появления во входном файле.

\OutputFile 

В выходной файл выведите два натуральных числа: размерность ячейки и ее номер. Если точка лежит во внешней области вывести -1.

\Examples

\begin{example}%
\exmp{
6 11 6
10 10 
10 -10
-10 10
-10 -10
5 - 5
-5 5
0 1
1 3
3 4
0 4
4 5
3 5 
5 0
2 5
2 3
2 0
4 1
3 7 9 6
3 3 6 4
3 7 5  8
3 2 4 5
3 1 2 10
3 0 3 10
4
0 0
5 5
-10 -10
100 100
}{
1 4
2 1
0 3
-1
}%
\end{example}

\end{problem}
