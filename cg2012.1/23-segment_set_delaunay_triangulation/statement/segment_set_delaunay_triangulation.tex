\begin{problem}{23. Триангуляция Делоне множества отрезков}{standard input}{standard output}{2 секунды}{256 мегабайт}

Пусть на плоскости задан планарный граф $G$. Триангуляцией называется планарный граф, все внутренние области которого являются треугольниками. Триангуляцией с ограничениями называется триангуляция, которая содержит в себе все ребра графа $G$. Говорят, что два смежных треугольника удовлетворяют условию Делоне, если для любого из двух треугольников противолежащая вершина другого треугольника не лежит внутри описанной окружности. Если в триангуляции с ограничениями
условие Делоне выполняется для всех пар смежных треугольников, которые не разделяются ребрами графа $G$, то она называтся триангуляцией Делоне графа $G$.

\InputFile

На первой строчке входного файла записано натуральное число $n$ --- количество точек. На последующих $n$ строчках через пробел записаны пары вещественных чисел $(x_i, y_i)$, задающих координаты точек на плоскости. На следующей строчке записано число $m$ --- количество отрезков. На последующих $m$ строчках через пробел записаны пары номеров точек начала и конца отрезка. Точки нумеруются с единицы в порядке появления в исходных данных. Гарантируется, что отрезки не пересекаются.

\OutputFile

В первой строчке выведите натуральное число $p$ --- число треугольников в триангуляции. На следующих $p$ строчках через пробел выведите номера точек в порядке обхода против часовой стрелки для каждого треугольника. Если триангуляций несколько, выведите любую из них.

\Examples

\begin{example}%
\exmp{
6
0 0
4 -2
8 0
2 2
3 1
4 2
1
1 3
}{
5
1 5 4
5 6 4
5 3 6
1 3 5
1 2 3
}%
\end{example}

\end{problem}
