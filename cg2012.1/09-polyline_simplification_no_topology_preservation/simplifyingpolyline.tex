\begin{problem}{09. Упрощение полигональной цепи}{douglaspeucker.in}{douglaspeucker.out}{2 секунды}{256 мегабайт}

Полигональная цепь — упорядоченное множество точек, последовательно соединенных отрезками.
Ваша задача — упростить полигональную цепь с помощью алгоритма Дугласа-Пеккера (\emph{Douglas-Peucker}) так,
чтобы расстояние от исходной цепи до получившейся было не больше $\varepsilon$.

\InputFile

На первой строке входного файла $n$ — количество вершин полигональной цепи.
На последующих $n$ строчках записаны пары вещественных чисел $(x_i, y_i)$, задающие координаты вершин.
В последней строке записано число $\varepsilon$.

\OutputFile

На первой строке выходного файла выведите $m$ — количество вершин в упрощенной полигональной цепи.
На второй строке через пробел выведите $m$ чисел — номера вершин, используемых в упрощенной полигональной цепи, в возрастающем порядке.
Нумерация точек начинается с единицы и задаётся в порядке следования их координат во входном файле.

\Examples

\begin{example}%
\exmp{
5
0 0
3 2
5 4
2 5
0 8
1.4142135
}{
3
1 3 5
}%
\end{example}
\Note
Пояснение к примеру:
	\begin{center}
		\includegraphics{picture.eps}
	\end{center}
\end{problem}