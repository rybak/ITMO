\begin{problem}{01. Выпуклая оболочка точек плоскости}{standard input}{standard output}{2 секунды}{256 мегабайт}

Выпуклой комбинацией множества точек $x_i$ называется линейная комбинация вида $\sum_i \alpha_i x_i$, где $\alpha_i$ — положительные вещественные числа, для которых $\sum_i \alpha_i = 1$. Все возможные выпуклые комбинации множества задают так называемую выпуклую оболочку. В данной задаче требуется найти выпуклую оболочку конечного множества точек, перечисленных во входном файле.

\InputFile

На первой строчке входного файла записано число $n$ — количество точек в множестве. На последующих $n$ строчках записаны пары вещественных чисел $(x_i, y_i)$, задающие координаты элементов множества на плоскости.

\OutputFile

Поскольку выпуклой оболочкой конечного множества точек на плоскости является выпуклый многоугольник (возможно, вырожденный), то для её задания достаточно перечислить точки, являющихся вершинами многоугольника.

На первой строке выходного файла выведите $m$ — количество вершин многоугольника. Количество точек должно быть минимальным. Далее, на $m$ строчках выведите координаты точек, являющихся вершинами многоугольника в порядке обхода против часовой стрелки. 

\Examples

\begin{example}%
\exmp{
5
0.0 0.0
0.0 1.0
1.0 0.0
1.0 1.0
0.5 0.5
}{
4
0.0 0.0
1.0 0.0
1.0 1.0
0.0 1.0
}%
\end{example}

\end{problem}
