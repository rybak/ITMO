\begin{problem}{09. Упрощение полигональной цепи}{standard input}{standard output}{2 секунды}{256 мегабайт}

Полигональная цепь — упорядоченное множество точек, последовательно соединенных отрезками.
Ваша задача — упростить полигональную цепь с помощью алгоритма Дугласа-Пеккера (\emph{Douglas-Peucker}) так,
чтобы расстояние от исходной цепи до получившейся было не больше $\varepsilon$.

\InputFile

На первой строке входного файла $n$ — количество вершин полигональной цепи.
На последующих $n$ строчках записаны пары вещественных чисел $(x_i, y_i)$, задающие координаты вершин.
В последней строке записано число $\varepsilon$.

\OutputFile

На первой строке выходного файла выведите $m$ — количество вершин в упрощенной полигональной цепи.
На второй строке через пробел выведите $m$ чисел — номера вершин, используемых в упрощенной полигональной цепи, в возрастающем порядке.
Нумерация точек начинается с единицы и задаётся в порядке следования их координат во входном файле.

\Examples

\begin{example}%
\exmp{
5
0 0
3 2
5 4
2 5
0 8
1.4142135
}{
3
1 3 5
}%
\end{example}
\Note
Пояснение к примеру:
	\begin{center}
	\tikzstyle{segment}=[draw,line width=2,-,black]
	\tikzstyle{simple}=[draw,line width=2,dashed,-,black!50]
	\tikzstyle{vertex}=[circle,fill=black,minimum size=5,inner sep=0pt]
	\begin{figure}[htb]
		\begin{minipage}[b]{1.0\linewidth}
		\begin{tikzpicture}
			\node(A1) at (0,0) 		[left] {$1$}; % label.lft (btex 1 etex, A[0]);
			\node(A2) at (3cm, 2cm) [below right] {$2$}; % label.lrt (btex 2 etex, A[1]);
			\node(A3) at (5cm, 4cm) [right] {$3$}; % label.rt (btex 3 etex, A[2]);
			\node(A4) at (2cm, 5cm) [below left] {$4$}; % label.llft (btex 4 etex, A[3]);
			\node(A5) at (8cm, 0cm) [left] {$5$}; % label.lft (btex 5 etex, A[4]);
% Опускаем перпендикуляр из точек A[1] и A[3] на A[0]--A[2] и A[2]--A[4] соответственно
%	pair B, H, C, T;
	% H и T - основания перпедикуляров
%	B := A[1] + (A[2] - A[0]) rotated 90;
%	H = whatever [B, A[1]];
%	H = whatever [A[0], A[2]];
%	C := A[3] + (A[2] - A[4]) rotated 90;
%	T = whatever [C, A[3]];
%	T = whatever [A[4], A[2]];
%   draw A[1]--H withcolor (red) withpen pencircle scaled 1.5 dashed evenly;
%	draw A[3]--T withcolor (red) withpen pencircle scaled 1.5 dashed evenly;

	% Цепь
%	draw A[0]--A[1]--A[2]--A[3]--A[4] withpen pencircle scaled 2;
	% Упрощенная цепь
%  	draw A[0]--A[2]--A[4] dashed evenly withpen pencircle scaled 2;

	% Линии от точек к осям
%	pickup pencircle scaled 1.2;
%	for i=0 step 1 until 4:
%		draw A[i]--(xpart A[i], 0) dashed evenly withcolor(.5white);
%		draw A[i]--(0, ypart A[i]) dashed evenly withcolor(.5white);
%	endfor;

		    % Координатные оси
			\def \xmax {7} % xmax := 7u;
			\def \ymax {9} % ymax := 9u;
			\draw [<->,thick] (0, \xmax) node (yaxis) [above] {$y$}
				|- (\ymax,0) node (xaxis) [right] {$x$};
			%\draw [->] (0, 0) -- (\xmax, 0) node[right] {$x$}; % drawarrow (0,0)--(xmax, 0);
			%\draw [->] (0, 0) -- (0, \ymax) node[above] {$y$}; % drawarrow (0,0)--(0, ymax);
			
			\foreach \point in {A1, A2, A3, A4, A5}
			{	
				\draw[dashed, grey] (yaxis |- \point) node[left] {$y'$} -| (xaxis -| \point) node[below] {$x'$};
			}
			\def \mark {0.3}
			\foreach \x in {1,...,\xmax}
			{
				\draw[-] (\x cm, -\mark cm) -- (\x cm, \mark cm); % draw (i,u/20)--(i,-u/20);
			}
			\foreach \y in {1, ...,\ymax} % for i=0 step u until ymax-u:
			{
				\draw[-] (-\mark cm, \y cm) -- (\mark cm, \y cm); % draw (u/20,i)--(-u/20,i);
			} % endfor;

			% SAMPLES

			\draw[dashed, green] (3.33, -1) -- (3.33, 2.5);
			\draw[-, green] (4.3, -1) -- (4.3, 2.5);
			\draw[dotted, blue] (10, -1) -- (10, 2.5);
			
			\draw[->, gray] (4.87, 2.4) to[out=-135, in=-45] (4, 2.4);

		 \end{tikzpicture}
	  \end{minipage}
	\end{figure}
	\end{center}
\end{problem}