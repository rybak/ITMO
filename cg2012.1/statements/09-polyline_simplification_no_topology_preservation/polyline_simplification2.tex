\begin{problem}{09. Упрощение полигональной цепи}{standard input}{standard output}{2 секунды}{256 мегабайт}

Полигональная цепь — упорядоченное множество точек, последовательно соединенных отрезками.
Ваша задача — упростить полигональную цепь с помощью алгоритма Дугласа-Пеккера (\emph{Douglas-Peucker}) так,
чтобы расстояние от исходной цепи до получившейся было не больше $\varepsilon$.

\InputFile

На первой строке входного файла $n$ — количество вершин полигональной цепи.
На последующих $n$ строчках записаны пары вещественных чисел $(x_i, y_i)$, задающие координаты вершин.
В последней строке записано число $\varepsilon$.

\OutputFile

На первой строке выходного файла выведите $m$ — количество вершин в упрощенной полигональной цепи.
На второй строке через пробел выведите $m$ чисел — номера вершин, используемых в упрощенной полигональной цепи, в возрастающем порядке.
Нумерация точек начинается с единицы и задаётся в порядке следования их координат во входном файле.

\Examples

\begin{example}%
\exmp{
5
0 0
3 2
5 4
2 5
0 8
1.4142135
}{
3
1 3 5
}%
\end{example}
\Note
Пояснение к примеру:
	\begin{center}
	\begin{figure}[htb]
%		\begin{minipage}[b]{1.0\linewidth}
		\begin{tikzpicture}
			\tikzstyle{segment}=[draw,line width=2,-,black]
			\tikzstyle{simple}=[draw,line width=2,dashed,-,black!50]
			\tikzstyle{vertex}=[circle,fill=black,minimum size=5,inner sep=0pt]

			\coordinate[label=left:$1$]			(A1) at (0, 0);
			\coordinate[label=below right:$2$]	(A2) at (3cm, 2cm);
			\coordinate[label=right:$3$]		(A3) at (5cm, 4cm);
			\coordinate[label=below left:$4$]	(A4) at (2cm, 5cm);
			\coordinate[label=left:$5$]			(A5) at (0, 8cm);
			
			\coordinate[label=above:$y$]		(yaxis) at (0, 9cm);
			\coordinate[label=above:$x$]		(xaxis) at (6cm, 0cm);
			\coordinate (zero) at (0cm, 0cm);
% Опускаем перпендикуляр из точек A[1] и A[3] на A[0]--A[2] и A[2]--A[4] соответственно
%	pair B, H, C, T;
	% H и T - основания перпедикуляров
%	B := A[1] + (A[2] - A[0]) rotated 90;
%	H = whatever [B, A[1]];
%	H = whatever [A[0], A[2]];
%	C := A[3] + (A[2] - A[4]) rotated 90;
%	T = whatever [C, A[3]];
%	T = whatever [A[4], A[2]];
%   draw A[1]--H withcolor (red) withpen pencircle scaled 1.5 dashed evenly;
%	draw A[3]--T withcolor (red) withpen pencircle scaled 1.5 dashed evenly;
			% TODO use pgfmath to calc H and T
			\draw[red] (A2) -- ($(A1)!(A2)!(A3)$);
			\draw[red] (A4) -- ($(A3)!(A4)!(A5)$);

			\foreach \point in {A1, A2, A3, A4, A5}
			{	
				\draw[dashed, black!50] (yaxis |- \point) -| (xaxis -| \point);
			}
			\draw [->,thick] (zero)--(yaxis);
			\draw [->,thick] (zero)--(xaxis);
			
			\def \mark {0.1}
			\foreach \x in {1,...,5}
			{
				\draw[-] (\x cm, -\mark cm) -- (\x cm, \mark cm); % draw (i,u/20)--(i,-u/20);
			}
			\foreach \y in {1,...,8} % for i=0 step u until ymax-u:
			{
				\draw[-] (-\mark cm, \y cm) -- (\mark cm, \y cm); % draw (u/20,i)--(-u/20,i);
			} % endfor;

			\draw[-,black, very thick] (A1) -- (A2) -- (A3) -- (A4) -- (A5);
			\draw[dashed, black, very thick] (A1) -- (A3) -- (A5);

		 \end{tikzpicture}
%	  \end{minipage}
	\end{figure}
	\end{center}
\end{problem}