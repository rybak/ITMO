\begin{problem}{11. Диаметр множества точек плоскости}{standard input}{standard output}{2 секунды}{256 мегабайт}

Диаметром $D$ множества точек $X$ называется максимальное попарное расстояние между всеми различными точками этого множества:
$$ D = \max\limits_{a, b \in X} \operatorname{distance}(a, b) $$

\InputFile

На первой строчке входного файла записано число $n$ — количество точек в множестве. На последующих $n$ строчках через пробел записаны пары вещественных чисел $(x_i, y_i)$, задающие координаты точек $X_i$ на плоскости.

\OutputFile
В выходной файл выведите координаты двух точек, принадлежащих заданному множеству, расстояние между которыми равно диаметру. Если таких пар точек несколько, выведите любую.

\Examples

\begin{example}%
\exmp{
5
-1 1
2 2
3 0
0 -1
-2 3
}{
-2 3
3 0
}%
\end{example}

\begin{example}
\exmp{
4
-2 -1
3 2
2 -2
-1 3
}{
-1 3
2 -2   
}%
\end{example}

Примечание: ответ “-2 -1 3 2” также верен.

\end{problem}
