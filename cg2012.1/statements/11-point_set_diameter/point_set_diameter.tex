\begin{problem}{11. Диаметр множества точек плоскости}{standard input}{standard output}{2 секунды}{256 мегабайт}

Диаметром множества точек $X$ называется максимальное попарное расстояние между всеми различными точками этого множества.
$$ D = \max\limits_{a, b \in X} \operatorname{distance}(a, b) $$.

\InputFile

На первой строчке входного файла записано число $n \le 50000$ — количество точек в множестве. На последующих $n$ строчках записаны пары вещественных чисел $(x_i, y_i), |x_i| , |y_i| \le 10000$, задающие координаты точек $X_i$ на плоскости.

\OutputFile
Для того, чтобы не выводить диаметр явно, достаточно вывести такие индексы $m$ и $k$, что расстояние между $X_m$ и $X_k$ максимально.

В единственной строке выходного файла выведите пару таких индексов (индексация — с 1).

\Examples

\begin{example}%
\exmp{
5
-1 1
2 2
3 0
0 -1
-2 3
}{
5 3
}%
\end{example}

\begin{example}
\exmp{
4
-2 -1
3 2
2 -2
-1 3
}{
1 2    
}%
\end{example}

Примечание: ответ “3 4” также верен.

\end{problem}
