\begin{problem}{11. Диаметр множества точек плоскости}{standard input}{standard output}{2 секунды}{256 мегабайт}

Диаметром $D$ множества точек $X$ называется максимальное попарное расстояние между всеми различными точками этого множества.
$$ D = \max\limits_{a, b \in X} \operatorname{distance}(a, b) $$.

\InputFile

На первой строчке входного файла записано число $n \le 50000$ — количество точек в множестве. На последующих $n$ строчках через пробел записаны пары вещественных чисел $(x_i, y_i), |x_i| , |y_i| \le 10000$, задающие координаты точек $X_i$ на плоскости.

\OutputFile
В выходной файл выведите координаты точек, расстояние между которыми равно диаметру. Если таких точек несколько, выберите минимальную пару.

Пара точек $(a, b)$ меньше пары точек $(c, d)$, если \\
$x_a < x_b$, или \\
$x_a = x_b$ и $y_a < y_b$, или \\
$x_a = x_b$ и $y_a = y_b$ и $x_c < x_d$, или \\
$x_a = x_b$ и $y_a = y_b$ и $x_c = x_d$ и $y_c < y_d$, \\
где между каждой парой координат — вещественное сравнение.

\Examples

\begin{example}%
\exmp{
5
-1 1
2 2
3 0
0 -1
-2 3
}{
-2 3
3 0
}%
\end{example}

\begin{example}
\exmp{
4
-2 -1
3 2
2 -2
-1 3
}{
-2 -1
3 2    
}%
\end{example}

Примечание: ответ “-1 3 2 -2” также верен, но он больше ответа “-2 -1 3 2”.

\end{problem}
