\begin{problem}{16. Пересечение множества окружностей}{standard input}{standard output}{2 секунды}{256 мегабайт}

На плоскости задано $n$ окружностей. Требуется вывести все точки пресечения этих окружностей.

\InputFile

На первой строчке входного файла записано число $n$ --- количество окружностей. На последующих $n$ строчках через пробел записаны три вещественных чисел $x_i, y_i, R_i$ --- координаты центра и радиус $i$-ой окружности соответственно.

\OutputFile

В выходной файл выведите число $m$ --- количество точек пересечения. На последующих $m$ строчках через пробел выведите пары вещественных чисел $(x_i, y_i)$ --- координаты точек пересечения

\Examples

\begin{example}%
\exmp{
5
0 0 1
0 -3 2
1 -1 1
5 1 3
-2 2 1
}{
4
1 0
0 -1
1 -2
2 -1
}%
\end{example}

\end{problem}
