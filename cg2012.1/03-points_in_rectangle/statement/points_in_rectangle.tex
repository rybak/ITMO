\begin{problem}{03. Поиск множества точек плоскости, попавших в прямоугольник}{standart input}{standart output}{2 секунды}{256 мегабайт}

На плоскости задано множество из $n$ точек. Также задано $m$ прямоугольников, стороны которых параллельны осям координат. Для каждого прямоугольника требуется вывести все точки множества, которые он содержит.

\InputFile

В первой строчке входного файла записано натуральное число $n$ --- количество точек в множестве. На последующих $n$ строчках через пробел записаны пары вещественных чисел $(x_i, y_i)$, задающих координаты точек на плоскости.
На следующей строчке записано натуральное число $m$ --- количество прямоугольников. На последующих $m$ строчках записаны по две пары вещественных чисел $(a_{x_i}, a_{y_i}, b_{x_i}, b_{y_i})$, задающих координаты левого нижнего угла и правого верхнего угла $i$-го прямоугольника соответственно.

\OutputFile

В выходной файл выведите $m$ блоков. В первой строчке каждого блока выведите число $c_i$ --- количество точек, попавших в $i$-ый прямоугольник. В следующие $c_i$ строчек выведите координаты этих точек.

\Examples

\begin{example}%
\exmp{
8
1 -1
2 3
4 5
4 2
-1 2
-2 3
0 3
-2 -1
4
-3 - 3 5 6
2 0 3 1
2 2 4 5
-2 -1 0 3
}{
8
1 -1 
2 3
4 5
4 2
-1 2
-2 3
0 3
-2 -1
0
3
2 3
4 5
4 2
4
-1 2
-2 3
0 3
-2 -1
}%
\end{example}

\end{problem}
