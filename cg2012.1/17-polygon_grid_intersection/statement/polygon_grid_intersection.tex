\begin{problem}{17. Пересечение многоугольника с сеткой}{standart input}{standart output}{2 секунды}{256 мегабайт}

На плоскости задан многоугольник. Требуется найти все точки пересечения мноугольника с прямыми вида $x = a$, $y = b$, где $a, b$ --- целые числа. При наложении ребра на прямую требуется вывести соответствующие ему вершины многоугольника.

\InputFile

В первой строчке входного файла записано натуральное число $n$ --- количество вершин многоугольника. В последующих $n$ строчках через пробел записаны пары вещественных чисел $(x_i, y_i)$, задающих координаты вершин на плоскости. Вершины обходятся против часовой стрелки.

\OutputFile

В первой строчке выходного файла выведите целое неотрицательное число $m$ --- количество точек пересечения. В последующих $m$ строчках выведите пары вещественных чисел $(x_i, y_i)$, задающих координаты пересечения многоугольника с прямыми. Точки обходятся в порядке обхода многоугольника.

\Examples

\begin{example}%
\exmp{
5
2 1
4 2
5 4
3 5
1 3
}{
11
2 1
3 0.3333
4 0.6667
5 2
5 3
5 4
4 4.5
3 5
2 4
1 3
1.5 2
}%
\exmp{
3
0.1 0.1
0.9 0.1
0.1 0.9
}{
0
}%
\end{example}

\end{problem}
