\begin{problem}{07. Растеризация выпуклого многоугольника}{standard input}{standard output}{2 секунды}{256 мегабайт}

Пусть на $\mathbb R^2$ задана координатная сетка: плоскость разбита на непересекающиеся ячейки вида $[n; n + 1) \times [m; m + 1)$, где $n$ и $m$ --- целые числа. Растеризацией объекта назовём множество ячеек, перечисленных в лексикографическом порядке, которые лежат внутри или на границе объекта. В данной задаче требуется найти растеризацию выпуклого многоугольника.

\InputFile

На первой строчке входного файла записано натуральное число $n$ --- количество вершин многоугольника. На последующих $n$ строчках через пробел записаны пары вещественных чисел $(x_i, y_i)$, задающих координаты вершин на плоскости. Вершины обходятся против часовой стрелки.

\OutputFile

В первой строчке выходных данных требуется вывести количество ячеек плоскости, которые участвуют в растеризации. Далее требуется на каждой строчке вывести ячейки плоскости (ячейка задаётся координатами нижнего левого угла) в лексикографическом порядке.

\Examples

\begin{example}%
\exmp{
4
0.5 0.5
4.5 0.5
4.5 2.5
2.5 2.5
}{
12
0 0
1 0
1 1
2 0
2 1
2 2
3 0
3 1
3 2
4 0
4 1
4 2
}%
\end{example}

\end{problem}
