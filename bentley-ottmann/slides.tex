\documentclass{beamer}
\usetheme{CambridgeUS}

%\documentclass[handout]{beamer}
%\usetheme{Pittsburgh}
%\beamertemplatesolidbackgroundcolor{black!2}
%\setbeamertemplate{footline}[frame number]
%\usepackage{pgfpages}
%%\pgfpagesuselayout{4 on 1}[a4paper,border shrink=5mm,landscape]
%\pgfpagesuselayout{8 on 1}[a4paper,border shrink=5mm]

%%% PACKAGES
\usepackage[russian]{babel}
\usepackage[utf8]{inputenc}
\usepackage{amsmath}
\usepackage{amssymb}
\usepackage{tikz}

%%% BEAMER SETTINGS
\setbeamertemplate{navigation symbols}{}
\setbeamertemplate{headline}{}

%%% TIKZ SETTINGS
\usetikzlibrary{fit}

%%% NEW COMMANDS
\def\pitem{\pause \item}

%\includeonlyframes{current} % leaves only the given frames

\title[Бентли-Оттман]{Пересечение множества отрезков, алгоритм Бентли-Оттмана}
\author[Андрей Шулаев]{Андрей Шулаев}
\institute[НИУ ИТМО]{Национальный исследовательский университет ИТ, механики и оптики}
\date{}

\begin{document}

%%%%%%%%%%%%%%%%%%%%%%%%%%%%%%%%%%%%%%%%%%%%%%%%%%%%%%%%%%%%%%%%%%%%%%%%%%%%%%%%%%%%%%%%%%%%%%%%%%%%%%%%%%%%%%%%%%%%%%%%
\frame[label=title]{\titlepage}

\begin{frame}{Постановка задачи}
\begin{block}{Задача}
Дано множество отрезков на плоскости, заданных координатами концов. Требуется найти все подможества отрезков, которые пересекаются в одной точке.
\end{block}
\end{frame}

\begin{frame}{Пример результата}
\begin{center}
\begin{tikzpicture}[scale=1.0]
  % коориднатная сетка
  \draw[step=1.0,gray,very thin,dashed] (-0.9,-0.9) grid (6.9,4.9);
  
  % координатные оси
  \draw[->] (0, -1) -- (0, 5);
  \draw[->] (-1, 0) -- (7, 0);
  
  % надписи на координатных осях
  \foreach \x in {1, 2, 3, 4, 5, 6}
    \draw (\x cm,1pt) -- (\x cm,-1pt) node[anchor=north] {\tiny{\x}};
  \foreach \y in {0, 1, 2, 3, 4}
    \draw (1pt,\y cm) -- (-1pt,\y cm) node[anchor=west] {\tiny{\y}};
  
  % отрезки
  \draw (1,5) node[anchor=south] {\scriptsize A} -- (2,2);
  \draw (1,3) node[anchor=north] {\scriptsize B} -- (3,5);
  \draw (2,1) node[anchor=north] {\scriptsize C} -- (5,3);
  \draw (3,1) node[anchor=north] {\scriptsize D} -- (4,3);
  \draw (4,1) node[anchor=north] {\scriptsize E} -- (3,3);
  \draw (4,5) -- (5,2) node[anchor=north] {\scriptsize F};
\end{tikzpicture}

Пересечения отрезков в примере: $\{\{A, B\}, \{C, D, E\}, \{C, F\}\}$
\end{center}
\end{frame}


\begin{frame}{Идея алгоритма}
\begin{itemize}
\pause \item Сканирующая прямая
\pause \item Три типа событий:
\begin{itemize}
\item Начало отрезка
\item Конец отрезка
\item Пересечение множества отрезков
\end{itemize}
\pause \item События обрабатываются по увеличению абсциссы
\pause \item В каждый момент времени поддерживается \emph{статус}: множество отрезков, которые пересекает сканирующая прямая, упорядоченное по ординате точки пересечения
\end{itemize}
\end{frame}

\begin{frame}{Пример работы}
\begin{center}
\begin{tikzpicture}[scale=0.8]
  % координатные оси
  \draw[->] (0, -1.2) -- (0, 6);
  \draw[->] (-1.2, 0) -- (9, 0);
  
  % надписи на координатных осях
  \foreach \x in {1, 2, 3, 4, 5, 6, 7, 8}
    \draw (\x cm,1pt) -- (\x cm,-1pt) node[anchor=north] {\tiny{\x}};
  \foreach \y in {0, 1, 2, 3, 4, 5}
    \draw (1pt,\y cm) -- (-1pt,\y cm) node[anchor=west] {\tiny{\y}};
  
  % отрезки
  \path[draw] (1,2) coordinate(A1) node[anchor=north] {\scriptsize A} -- (4, 3) coordinate(A2);
  \path[draw] (2,3) coordinate(B1) node[anchor=south] {\scriptsize B} -- (5, 1) coordinate(B2);
  \path[draw] (3,1) coordinate(C1) node[anchor=north] {\scriptsize C} -- (6, 4) coordinate(C2);
  \path[draw] (5,5) coordinate(D1) node[anchor=south] {\scriptsize D} -- (8, 4) coordinate(D2);
  
  % сканирующая прямая
  \only<2> { \draw[color=red,dashed] (0.5,-0.5) node[anchor=north] {\scriptsize $\varnothing$} -- +(0, 6); }
  \only<3> { \draw[color=red,dashed] (1.0,-0.5) node[anchor=north] {\scriptsize A} -- +(0, 6); }
  \only<4> { \draw[color=red,dashed] (2.0,-0.5) node[anchor=north] {\scriptsize A, B} -- +(0, 6); }
  \only<5> { \draw[color=red,dashed] (2.666,-0.5) node[anchor=north] {\scriptsize B, A} -- +(0, 6); }
  \only<6> { \draw[color=red,dashed] (3.0,-0.5) node[anchor=north] {\scriptsize C, B, A} -- +(0, 6); }
  \only<7> { \draw[color=red,dashed] (3.8,-0.5) node[anchor=north] {\scriptsize B, C, A} -- +(0, 6); }
  \only<8> { \draw[color=red,dashed] (5.0,-0.5) node[anchor=north] {\scriptsize B, C} -- +(0, 6); }
  \only<9-10> { \draw[color=red,dashed] (5.0,-0.5) node[anchor=north] {\scriptsize
    \only<9>{C}
    \only<10>{C, D}
  } -- +(0, 6); }
  \only<11> { \draw[color=red,dashed] (6.0,-0.5) node[anchor=north] {\scriptsize D} -- +(0, 6); }
  \only<12> { \draw[color=red,dashed] (8.0,-0.5) node[anchor=north] {\scriptsize $\varnothing$} -- +(0, 6); }
  \only<13> { \draw[color=red,dashed] (8.5,-0.5) node[anchor=north] {\scriptsize $\varnothing$} -- +(0, 6); }
  
  % точки пересечения
  \only<5-> { \fill[red] (intersection of A1--A2 and B1--B2) circle(2.0pt); }
  \only<7-> { \fill[red] (intersection of B1--B2 and C1--C2) circle(2.0pt); }
\end{tikzpicture}
\end{center}
\end{frame}

\begin{frame}{Хранение статуса}
\begin{itemize}
\item Статус требует поддержания следующих операций:
\begin{itemize}
\item Вставка в произвольную позицию
\item Переворот подотрезка
\end{itemize}
\pause
\item Как реализовать?
\pause
\begin{itemize}
\item Просто: список, операции за $O(n)$
\item Быстро: декартово дерево по неявному ключу, операции за $O(\log n)$
\end{itemize}
\end{itemize}
\end{frame}

\begin{frame}{Проблемы и решения}

\begin{itemize}
\pause \item Проблема вертикальных отрезков и наличия более одного события разрешается обработкой событий на текущей абсциссе в порядке увеличения ординаты
\end{itemize}
\pause

\begin{center}
\begin{tikzpicture}[scale=0.7]
  % координатные оси
  \draw[->] (0, -1.2) -- (0, 6);
  \draw[->] (-1.2, 0) -- (9, 0);
  
  % надписи на координатных осях
  \foreach \x in {1, 2, 3, 4, 5, 6, 7, 8}
    \draw (\x cm,1pt) -- (\x cm,-1pt) node[anchor=north] {\tiny{\x}};
  \foreach \y in {0, 1, 2, 3, 4, 5}
    \draw (1pt,\y cm) -- (-1pt,\y cm) node[anchor=west] {\tiny{\y}};
  
  % отрезки
  \path[draw] (1,2) coordinate(A1) node[anchor=north] {\scriptsize A} -- (4, 3) coordinate(A2);
  \path[draw] (2,3) coordinate(B1) node[anchor=south] {\scriptsize B} -- (5, 1) coordinate(B2);
  \path[draw] (3,1) coordinate(C1) node[anchor=north] {\scriptsize C} -- (6, 4) coordinate(C2);
  \path[draw] (5,5) coordinate(D1) node[anchor=south] {\scriptsize D} -- (8, 4) coordinate(D2);
  
  % сканирующая прямая
  \only<3-5> { \draw[color=red,dashed] (5.0,-0.5) node[anchor=north] {\scriptsize
    \only<3>{B, C}
    \only<4>{C}
    \only<5>{C, D}
  } -- +(0, 6); }
  
  % точки пересечения
  \fill[red] (intersection of A1--A2 and B1--B2) circle(2.0pt);
  \fill[red] (intersection of B1--B2 and C1--C2) circle(2.0pt);
\end{tikzpicture}
\end{center}

\end{frame}

\begin{frame}{Проблемы и решения}
\begin{itemize}
\pause \item Если в точке пересекается множество отрезков, то они идут подряд в статусе и после точки соответствующий отрезок надо инвертировать
\end{itemize}
\pause

\begin{center}
\begin{tikzpicture}[scale=0.7]
  % координатные оси
  \draw[->] (0, -1.2) -- (0, 6);
  \draw[->] (-1.2, 0) -- (9, 0);
  
  % надписи на координатных осях
  \foreach \x in {1, 2, 3, 4, 5, 6, 7, 8}
    \draw (\x cm,1pt) -- (\x cm,-1pt) node[anchor=north] {\tiny{\x}};
  \foreach \y in {0, 1, 2, 3, 4, 5}
    \draw (1pt,\y cm) -- (-1pt,\y cm) node[anchor=west] {\tiny{\y}};
  
  % отрезки
  \path[draw] (1,1) coordinate(A1) node[anchor=north] {\scriptsize A} -- (7,4) coordinate(A2);
  \path[draw] (2,1) coordinate(B1) node[anchor=north] {\scriptsize B} -- (6,4) coordinate(B2);
  \path[draw] (3,1) coordinate(C1) node[anchor=north] {\scriptsize C} -- (5,4) coordinate(C2);
  \path[draw] (1.5,2) coordinate(D1) node[anchor=south] {\scriptsize D} -- (5.5,2) coordinate(D2);
  
  % сканирующая прямая
  \only<3> { \draw[color=red,dashed] (3.8,-0.5) node[anchor=north] {\scriptsize D, C, B, A} -- +(0, 6); }
  \only<4> { \draw[color=red,dashed] (4.0,-0.5) node[anchor=north] {\scriptsize D, \textbf{C, B, A}} -- +(0, 6); }
  \only<5> { \draw[color=red,dashed] (4.0,-0.5) node[anchor=north] {\scriptsize D, A, B, C} -- +(0, 6); }
  \only<6> { \draw[color=red,dashed] (4.2,-0.5) node[anchor=north] {\scriptsize D, A, B, C} -- +(0, 6); }
  
  % точки пересечения
  \fill[red] (intersection of A1--A2 and D1--D2) circle(2.0pt);
  \fill[red] (intersection of B1--B2 and D1--D2) circle(2.0pt);
  \fill[red] (intersection of C1--C2 and D1--D2) circle(2.0pt);
  \fill[red] (intersection of A1--A2 and B1--B2) circle(2.0pt);
\end{tikzpicture}
\end{center}
\end{frame}

\begin{frame}{Оценки времени работы и используемой памяти}
\begin{itemize}
\item Упорядочение событий — $O(n\log n)$ времени, $O(n)$ памяти
\item Хранение статуса — $O(n\log n)$ времени, $O(n)$ памяти
\item Вывод ответа — $O(k)$ времени ($k$ — размер ответа).
\end{itemize}
\end{frame}

\begin{frame}{Ссылки}
\begin{itemize}
\item Кормен, Лейзерсон, Ривест, Штайн. <<Алгоритмы: построение и анализ>>, раздел <<Вычислительная геометрия>>.
\item Bentley, Ottman. <<Algorithms for Reporting and Counting Geometric Intersections>>.
\item Препарата, Шеймос. <<Вычислительная геометрия: введение>>.
\end{itemize}
\end{frame}

\begin{frame}
  \centerline{\huge{\textbf{Вопросы?}}}
\end{frame}

\end{document}
