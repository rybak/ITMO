\begin{problem}{Выпуклая оболочка}{standard input}{standard output}{2 секунды}{256 мегабайт}

Выпуклой комбинацией множества точек $x_i$ называется линейная комбинация вида $\sum_i \alpha_i x_i$, где $\alpha_i$ — положительные вещественные числа, для которых $\sum_i \alpha_i = 1$. Все возможные выпуклые комбинации множества задают так называемую выпуклую оболочку. В данной задаче требуется найти выпуклую оболочку конечного множества точек, перечисленных во входном файле.

\InputFile

На первой строчке входного файла записано число $n$ — количество точек в множестве. На последующих $n$ строчках записаны пары вещественных чисел $(x_i, y_i)$, задающие координаты элементов множества на плоскости.

\OutputFile

Поскольку выпуклой оболочкой конечного множества точек на плоскости является выпуклый многоугольник с вершинами в элементах множества, то для её задания достаточно перечислить номера элементов, являющихся вершинами многоугольника.

На первой строке выходного файла выведите $m$ — количество точек в многоугольнике. На второй строке через пробел выведите $m$ чисел — номера точек, являющихся вершинами выпуклой оболочки. Номера требуется вывести в порядке обхода против часовой стрелки. Нумерация точек начинается с единицы и задаётся в порядке следования их координат во входном файле.

\Examples

\begin{example}%
\exmp{
5
0.0 0.0
0.0 1.0
1.0 0.0
1.0 1.0
0.5 0.5
}{
4
1 2 4 3
}%
\end{example}

\end{problem}
