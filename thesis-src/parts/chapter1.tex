\chapter{Обзор предметной области}
\label{chapter1}

% 
% * Очень кратко о ФП и паттерн-матчинге.
% * Кратко о теории типов, отношении конвертабельности и Мартине-Лёфе.
% * Кратко об унификации и индексированных семествах.
% * Кратко об Агда.
% * Об известных штуках кодируемых индексируемыми семействами (ссылки из
%   статьи МакБрайда).
%
%
В программировании структуры данных позволяют упростить обработку
однотипных и/или логически связанных данных.
Задача структур данных — облегчить написание программ для программистов и
ускорить обработку данных.

\section{Структуры данных}
Структуры данных используются в программировании для абстрагирования
обработки связанных и однородных данных.

Часто используемые структуры данных включаются в стандартные библиотеки
языков программирования.

\subsection{Функциональные структуры данных}

Главное отличие функциональных структур данных от императивных \cite{OkasakiBook}
заключается в неиспользовании разрушающих обновлений (то есть присваиваний).
При обновлении структуры данных все измененные части создаются заново.

\section{Индуктивные семейства и зависимые типы}

\begin{definition}
\emph{Индуктивное семейство} \cite{DybjerIndFam}— это семейство типов данных,
которые могут зависеть от других типов и значений.

Тип или значение, от которого зависит зависимый тип, называют \emph{индексом}.
\end{definition}

Одной из областей применения индуктивных семейств являются системы интерактивного
доказательства теорем.

Индуктивные семейства позволяют формализовать математические структуры,
кодируя утверждения о структурах в них самих, тем самым перенося сложность из
доказательств в определения.

% \section{Существующие подходы}
В работах \cite{OkasakiThesis, McBridePivotal} приведены различные подходы
к построению функциональных структур данных.

Пример задания инвариантов для
структуры данных — тип данных для хранения баланса в AVL-дереве \cite{AVLTree}.
\newline
\AgdaHide{
\begin{code}\>\<%
\\
\>\AgdaKeyword{module} \AgdaModule{AVLBalance} \AgdaKeyword{where}\<%
\\
\>[0]\AgdaIndent{2}{}\<[2]%
\>[2]\AgdaKeyword{open} \AgdaKeyword{import} \AgdaModule{AgdaDescription}\<%
\\
\>[0]\AgdaIndent{2}{}\<[2]%
\>[2]\AgdaKeyword{infix} \AgdaNumber{4} \_∼\_\<%
\\
\>[0]\AgdaIndent{2}{}\<[2]%
\>[2]\AgdaFunction{testˡ} \AgdaSymbol{:} \AgdaDatatype{ℕ}\<%
\\
\>[0]\AgdaIndent{2}{}\<[2]%
\>[2]\AgdaFunction{testˡ} \AgdaSymbol{=} \AgdaInductiveConstructor{succ} \AgdaInductiveConstructor{zero}\<%
\\
\>\<\end{code}
}
Если $m \sim n$, то разница между $m$ и $n$ не больше чем один:
\begin{code}\>\<%
\\
\>[0]\AgdaIndent{2}{}\<[2]%
\>[2]\AgdaKeyword{data} \AgdaDatatype{\_∼\_} \AgdaSymbol{:} \AgdaDatatype{ℕ} \AgdaSymbol{→} \AgdaDatatype{ℕ} \AgdaSymbol{→} \AgdaPrimitiveType{Set} \AgdaKeyword{where}\<%
\\
\>[2]\AgdaIndent{4}{}\<[4]%
\>[4]\AgdaInductiveConstructor{∼+} \AgdaSymbol{:} \AgdaSymbol{∀} \AgdaSymbol{\{}\AgdaBound{n}\AgdaSymbol{\}} \AgdaSymbol{→} \<[21]%
\>[21]\AgdaBound{n} \AgdaDatatype{∼} \AgdaNumber{1} \AgdaFunction{+} \AgdaBound{n}\<%
\\
\>[2]\AgdaIndent{4}{}\<[4]%
\>[4]\AgdaInductiveConstructor{∼0} \AgdaSymbol{:} \AgdaSymbol{∀} \AgdaSymbol{\{}\AgdaBound{n}\AgdaSymbol{\}} \AgdaSymbol{→} \<[21]%
\>[21]\AgdaBound{n} \AgdaDatatype{∼} \AgdaBound{n}\<%
\\
\>[2]\AgdaIndent{4}{}\<[4]%
\>[4]\AgdaInductiveConstructor{∼-} \AgdaSymbol{:} \AgdaSymbol{∀} \AgdaSymbol{\{}\AgdaBound{n}\AgdaSymbol{\}} \AgdaSymbol{→} \AgdaNumber{1} \AgdaFunction{+} \AgdaBound{n} \AgdaDatatype{∼} \AgdaBound{n}\<%
\\
\>\<\end{code}


\section{Agda}
\textit{Agda}~\cite{AgdaLang}~---  чистый функциональный язык программирования с зависимыми типами.
В Agda есть поддержка модулей:
\begin{code}\>\<%
\\
\>\AgdaKeyword{module} \AgdaModule{AgdaDescription} \AgdaKeyword{where}\<%
\\
\>\<\end{code} В коде на Agda широко используются символы Unicode.
Тип натуральных чисел — \D{ℕ}.
\begin{code}\>\<%
\\
\>\AgdaKeyword{data} \AgdaDatatype{ℕ} \AgdaSymbol{:} \AgdaPrimitiveType{Set} \AgdaKeyword{where}\<%
\\
\>[0]\AgdaIndent{2}{}\<[2]%
\>[2]\AgdaInductiveConstructor{zero} \AgdaSymbol{:} \AgdaDatatype{ℕ}\<%
\\
\>[0]\AgdaIndent{2}{}\<[2]%
\>[2]\AgdaInductiveConstructor{succ} \AgdaSymbol{:} \AgdaDatatype{ℕ} \AgdaSymbol{→} \AgdaDatatype{ℕ}\<%
\\
\>\<\end{code} \AgdaHide{
\begin{code}\>\<%
\\
\>\AgdaSymbol{\{-\#} \AgdaKeyword{BUILTIN} NATURAL \AgdaDatatype{ℕ} \AgdaSymbol{\#-\}}\<%
\\
\>\AgdaSymbol{\{-\#} \AgdaKeyword{BUILTIN} ZERO \AgdaInductiveConstructor{zero} \AgdaSymbol{\#-\}}\<%
\\
\>\AgdaSymbol{\{-\#} \AgdaKeyword{BUILTIN} SUC \AgdaInductiveConstructor{succ} \AgdaSymbol{\#-\}}\<%
\\
\>\<\end{code}
}
В Agda функции можно определять как mixfix операторы.
Пример — сложение натуральных чисел:
\begin{code}\>\<%
\\
\>\AgdaFunction{\_+\_} \AgdaSymbol{:} \AgdaDatatype{ℕ} \AgdaSymbol{→} \AgdaDatatype{ℕ} \AgdaSymbol{→} \AgdaDatatype{ℕ}\<%
\\
\>\AgdaInductiveConstructor{zero} \AgdaFunction{+} \AgdaBound{b} \AgdaSymbol{=} \AgdaBound{b}\<%
\\
\>\AgdaInductiveConstructor{succ} \AgdaBound{a} \AgdaFunction{+} \AgdaBound{b} \AgdaSymbol{=} \AgdaInductiveConstructor{succ} \AgdaSymbol{(}\AgdaBound{a} \AgdaFunction{+} \AgdaBound{b}\AgdaSymbol{)}\<%
\\
\>\<\end{code}
Символы подчеркивания обозначают места для аргументов.
% Система типов \textit{Agda} позволяет ... 
% В отличие от \textit{Haskell}, в \textit{Agda} имеется ... 

Зависимые типы позволяют определять типы, зависящие (индексированные) от значений
других типов. Пример — список, индексированный своей длиной:
\begin{code}\>\<%
\\
\>\AgdaKeyword{data} \AgdaDatatype{Vec} \AgdaSymbol{(}\AgdaBound{A} \AgdaSymbol{:} \AgdaPrimitiveType{Set}\AgdaSymbol{)} \AgdaSymbol{:} \AgdaDatatype{ℕ} \AgdaSymbol{→} \AgdaPrimitiveType{Set} \AgdaKeyword{where}\<%
\\
\>[0]\AgdaIndent{2}{}\<[2]%
\>[2]\AgdaInductiveConstructor{nil} \<[7]%
\>[7]\AgdaSymbol{:} \AgdaDatatype{Vec} \AgdaBound{A} \AgdaInductiveConstructor{zero}\<%
\\
\>[0]\AgdaIndent{2}{}\<[2]%
\>[2]\AgdaInductiveConstructor{cons} \AgdaSymbol{:} \AgdaSymbol{∀} \AgdaSymbol{\{}\AgdaBound{n}\AgdaSymbol{\}} \AgdaSymbol{→} \AgdaBound{A} \AgdaSymbol{→} \AgdaDatatype{Vec} \AgdaBound{A} \AgdaBound{n} \AgdaSymbol{→} \AgdaDatatype{Vec} \AgdaBound{A} \AgdaSymbol{(}\AgdaInductiveConstructor{succ} \AgdaBound{n}\AgdaSymbol{)}\<%
\\
\>\<\end{code}
В фигурные скобки заключаются неявные аргументы.

Такое определение позволяет нам описать функцию $ \F{head} $ для такого списка, которая не может бросить исключение:
\begin{code}\>\<%
\\
\>\AgdaFunction{head} \AgdaSymbol{:} \AgdaSymbol{∀} \AgdaSymbol{\{}\AgdaBound{A}\AgdaSymbol{\}} \AgdaSymbol{\{}\AgdaBound{n}\AgdaSymbol{\}} \AgdaSymbol{→} \AgdaDatatype{Vec} \AgdaBound{A} \AgdaSymbol{(}\AgdaInductiveConstructor{succ} \AgdaBound{n}\AgdaSymbol{)} \AgdaSymbol{→} \AgdaBound{A}\<%
\\
\>\<\end{code}
У аргумента функции $ \F{head} $ тип $ \D{Vec}\,A\,(\DC{succ}\,n) $, то есть вектор, в котором есть хотя бы один элемент.
Это позволяет произвести сопоставление с образцом только по конструктору $ \DC{cons} $:
\begin{code}\>\<%
\\
\>\AgdaFunction{head} \AgdaSymbol{(}\AgdaInductiveConstructor{cons} \AgdaBound{a} \AgdaBound{as}\AgdaSymbol{)} \AgdaSymbol{=} \AgdaBound{a}\<%
\\
\>\<\end{code}



\section{Выводы по главе \ref{chapter1}}
Рассмотрены некоторые существующие подходы к построению структур данных
с использованием индуктивных семейств.
Кратко описаны особенности языка программирования \textit{Agda}.
