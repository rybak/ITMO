\chapter{Обзор предметной области}
\label{chapter1}

В <...> структуры данных позволяют хранить и обрабатывать множество однотипных 
и/или логически связанных данных в вычислительной технике.
Задача (?) структур данных — облегчить написание программ для программистов и
ускорить обработку данных.

\section{Структуры данных}
Структуры данных используются в программировании для абстрагирования
обработки связанных и однородных данных.

% Существуют различные виды . 

Часто используемые структуры данных ... включаются в стандартные библиотеки
языков программирования.
Существует несколько различных .... Основные из них:
\begin{itemize}
 \item foobar
\end{itemize}

\subsection{Функциональные структуры данных}
...
В отличие от ...

% Есть Оказаки с его чистофункциональными структурами данных.


\section{Индуктивные семейства и зависимые типы}

Индуктивные семейства — это типы данных, которые могут зависеть от типов и значений. Например, тип векторов индексированных длиной 

\begin{code}\>\<%
\\
\>\AgdaKeyword{module} \AgdaModule{VecSample} \AgdaKeyword{where}\<%
\\
\>\AgdaKeyword{data} \AgdaDatatype{ℕ} \AgdaSymbol{:} \AgdaPrimitiveType{Set} \AgdaKeyword{where}\<%
\\
\>[0]\AgdaIndent{2}{}\<[2]%
\>[2]\AgdaInductiveConstructor{zero} \AgdaSymbol{:} \AgdaDatatype{ℕ}\<%
\\
\>[0]\AgdaIndent{2}{}\<[2]%
\>[2]\AgdaInductiveConstructor{succ} \AgdaSymbol{:} \AgdaDatatype{ℕ} \AgdaSymbol{→} \AgdaDatatype{ℕ}\<%
\\
\>\AgdaKeyword{data} \AgdaDatatype{Vec} \AgdaBound{A} \AgdaSymbol{:} \AgdaDatatype{ℕ} \AgdaSymbol{→} \AgdaPrimitiveType{Set} \AgdaKeyword{where}\<%
\\
\>[0]\AgdaIndent{2}{}\<[2]%
\>[2]\AgdaInductiveConstructor{nil} \<[7]%
\>[7]\AgdaSymbol{:} \AgdaDatatype{Vec} \AgdaBound{A} \AgdaInductiveConstructor{zero}\<%
\\
\>[0]\AgdaIndent{2}{}\<[2]%
\>[2]\AgdaInductiveConstructor{cons} \AgdaSymbol{:} \AgdaSymbol{∀} \AgdaSymbol{\{}\AgdaBound{n}\AgdaSymbol{\}} \AgdaSymbol{→} \AgdaBound{A} \AgdaSymbol{→} \AgdaDatatype{Vec} \AgdaBound{A} \AgdaBound{n} \AgdaSymbol{→} \AgdaDatatype{Vec} \AgdaBound{A} \AgdaSymbol{(}\AgdaInductiveConstructor{succ} \AgdaBound{n}\AgdaSymbol{)}\<%
\\
\>\<\end{code}

Такое определение позволяет нам описать функцию $ \F{head} $ для такого списка, которая не может бросить исключение:
\begin{code}\>\<%
\\
\>\AgdaFunction{head} \AgdaSymbol{:} \AgdaSymbol{∀} \AgdaSymbol{\{}\AgdaBound{A}\AgdaSymbol{\}} \AgdaSymbol{\{}\AgdaBound{n}\AgdaSymbol{\}} \AgdaSymbol{→} \AgdaDatatype{Vec} \AgdaBound{A} \AgdaSymbol{(}\AgdaInductiveConstructor{succ} \AgdaBound{n}\AgdaSymbol{)} \AgdaSymbol{→} \AgdaBound{A}\<%
\\
\>\<\end{code}
У аргумента функции $ \F{head} $ тип $ \D{Vec}\,A\,(\DC{succ}\,n) $, то есть вектор, в котором есть хотя бы один элемент.
Это позволяет произвести сопоставление с образцом только по конструктору $ \DC{cons} $:
\begin{code}\>\<%
\\
\>\AgdaFunction{head} \AgdaSymbol{(}\AgdaInductiveConstructor{cons} \AgdaBound{a} \AgdaBound{as}\AgdaSymbol{)} \AgdaSymbol{=} \AgdaBound{a}\<%
\\
\>\<\end{code}


Одной из областей применения индуктивных семейств являются системы интерактивного
доказательства теорем.

Индуктивные семейства позволяют формализовать математические структуры,
кодируя утверждения о структурах в них самих, тем самым перенося сложность из
доказательств в определения.
% \section{Существующие подходы}

\section{Agda}
\textit{Agda}~\cite{AgdaLang}~---  чистый функциональный язык программирования с зависимыми типами.
В Agda есть индуктивные семейства.
В Agda также есть параметризованные модули, mixfix операторы,
В коде на Agda широко используются символы Unicode.

В фигурных скобках — неявные аргументы, которые 

% Система типов \textit{Agda} позволяет ... 
% В отличие от \textit{Haskell}, в \textit{Agda} имеется ... 

\section{Выводы по главе \ref{chapter1}}
Рассмотрены некоторые существующие подходы к ....
Описаны различные ....
Кратко описана ... \textit{...}.
Кратко описаны особенности языка программирования \textit{Agda}.


