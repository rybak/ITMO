\documentclass[a4paper]{report}
%uncomment to see the references
%\usepackage{showkeys}

\usepackage{amsmath, amssymb, amsthm}
\usepackage{amsfonts, amsxtra}
% \usepackage{bbm}
\usepackage[english,russian]{babel}

% This handles the translation of unicode to latex:

\usepackage[utf8]{inputenc}
% \usepackage{autofe}
\usepackage[T2A]{fontenc}

\usepackage[section]{algorithm}
\usepackage{algorithmic}

\usepackage[backend=biber,bibencoding=utf8,sorting=none,sortcites=true,bibstyle=sty/gost71,maxnames=99,citestyle=numeric-comp,babel=other]{biblatex}

\defbibenvironment{bibliography}
  {\list
     {\printfield[labelnumberwidth]{labelnumber}.}
     {\setlength{\labelwidth}{2\labelnumberwidth}%
      \setlength{\leftmargin}{\labelwidth}%
      \setlength{\labelsep}{\biblabelsep}%
      \addtolength{\leftmargin}{\labelsep}%
      \setlength{\itemsep}{\bibitemsep}%
      \setlength{\parsep}{\bibparsep}}%
      \renewcommand*{\makelabel}[1]{\hss##1}}
  {\endlist}
  {\item}

% \usepackage{csquotes}
%\usepackage{expdlist}
%\usepackage[nottoc,notbib]{tocbibind}
\usepackage[pdftex]{graphicx}
\graphicspath{{pic/}}
\usepackage{sty/dbl12}
\usepackage{epsfig}
% \usepackage{verbatim}
\usepackage{sty/rac}
% \usepackage{listings}
\usepackage{agda}
\usepackage[singlelinecheck=false]{caption}

% \usepackage{fancyvrb}
% \DefineVerbatimEnvironment
%   {code}{Verbatim}
%   {} % Add fancy options here if you like.

% Some characters that are not automatically defined
% (you figure out by the latex compilation errors you get),
% and you need to define:

\DeclareUnicodeCharacter{8988}{\ensuremath{\ulcorner}}
\DeclareUnicodeCharacter{8989}{\ensuremath{\urcorner}}
\DeclareUnicodeCharacter{8803}{\ensuremath{\overline{\equiv}}}

%%%%%%%%%%%%%%%%%%%%%%%%%%%%%%%%%%%%%%%%%%%%%%%%%%%%%%%%%%%%%%%%%%%%%%%%%%%%%%

\captionsetup[figure]{justification=centering,   position=bottom, skip=0pt}
\captionsetup[table] {justification=raggedright, position=top,    skip=0pt}

% % Redefine margins and other page formatting
% 
% \setlength{\oddsidemargin}{0.5in}

% Various theorem environments. All of the following have the same numbering
% system as theorem.

\theoremstyle{plain}
\newtheorem{theorem}{Теорема}
\newtheorem{prop}[theorem]{Утверждение}
\newtheorem{corollary}[theorem]{Следствие}
\newtheorem{lemma}[theorem]{Лемма}
\newtheorem{question}[theorem]{Вопрос}
\newtheorem{conjecture}[theorem]{Гипотеза}
\newtheorem{assumption}[theorem]{Предположение}

\theoremstyle{definition}
\newtheorem{definition}[theorem]{Определение}
\newtheorem{notation}[theorem]{Обозначение}
\newtheorem{condition}[theorem]{Условие}
\newtheorem{example}[theorem]{Пример}
%\newtheorem{algorithm}[theorem]{Алгоритм}
\floatname{algorithm}{Листинг}
\renewcommand{\algorithmicrequire}{\textbf{Вход:}}

%\newtheorem{introduction}[theorem]{Introduction}

\renewcommand{\proof}{\\\textbf{Доказательство.}~}
% \renewcommand{\lstlistingname}{Листинг}
 
% \lstnewenvironment{snippet}[1][]%
% {
%    \noindent
%    \minipage{\linewidth} 
%    \vspace{0.5\baselineskip}
%    \lstset{basicstyle=\ttfamily\footnotesize,frame=single,#1}}
% {\endminipage}

%%%%%%%%%%%%%%%%%%%%%%%%%%%%%%%%%%%%%%%%%%%%%%%%%%%%%%%%%%%%%%%%%%%%%%%%%%%%%%%

\numberwithin{theorem}{chapter}        % Numbers theorems "x.y" where x
                                        % is the section number, y is the
                                        % theorem number

%\renewcommand{\thetheorem}{\arabic{chapter}.\arabic{theorem}}

%\makeatletter                          % This sequence of commands will
%\let\c@equation\c@theorem              % incorporate equation numbering
%\makeatother                           % into the theorem numbering scheme

%\renewcommand{\theenumi}{(\roman{enumi})}

%%%%%%%%%%%%%%%%%%%%%%%%%%%%%%%%%%%%%%%%%%%%%%%%%%%%%%%%%%%%%%%%%%%%%%%%%%%%%%

\binoppenalty=10000
\relpenalty=10000

\addbibresource{thesis.bib}

\begin{document}
%  \renewcommand{\thelstlisting}{\thesection.\arabic{lstlisting}}
% Begin the front matter as required by Rackham dissertation guidelines
\initializefrontsections

\pagestyle{title}

\begin{center}
Санкт-Петербургский национальный исследовательский университет \\ информационных технологий, механики и оптики

\vspace{2cm}

Факультет информационных технологий и программирования

Кафедра компьютерных технологий

\vspace{3cm}

{\Large Рыбак Андрей Викторович}

\vspace{2cm}

\vbox{\LARGE\bfseries
Представление структур данных индуктивными семействами и доказательства их свойств
}

\vspace{4cm}

{\Large Научный руководитель: ассистент кафедры ТП Я.~М.~Малаховски}

\vspace{6cm}

Санкт-Петербург\\ 2014
\end{center}

\newpage

\setcounter{page}{3}
\pagestyle{plain}

\tableofcontents
%\listoffigures

% Chapters
\startthechapters
% \startprefacepage

Структуры данных используются в программировании повсеместно для
упрощения хранения и обработки данных.

Практика показывает, что тривиальные структуры данных хорошо выражаются в форме
индуктивных семейств.
Мы хотим узнать насколько хорошо эта практика работает и для более сложных
структур.

\iffalse
hello world 
\newline
\fi
Чисто функциональные структуры данных.


\chapter{Обзор
% предметной области
}
\label{chapter1}

В данной главе производится обзор предметной области и даются определения используемых терминов.

\section{Функциональное программирование}

\emph{Функциональное программирование} — парадигма программирования,
являющаяся разновидностью декларативного программирования,
в которой программу представляют в виде функций
(в математическом смысле этого слова, а не в смысле, используемом в процедурном программировании),
а выполнением программы считают вычисление значений применения этих функций к заданным значениям.
Большинство функциональных языков программирования используют в своём основании лямбда-исчисление
(например, Haskell~\cite{HaskellLang}, Curry~\cite{CurryLang}, Agda~\cite{AgdaLang},
диалекты LISP~\cite{SchemeLang,ClojureLang,SICP}, SML~\cite{SMLLang}, OCaml\cite{OCamlLang}),
но существуют и функциональные языки явно не основанные на этом формализме
(например, препроцессор языка C и шаблоны в C++).

\section{Лямбда-исчисление}

\emph{Лямбда-исчисление} ($\lambda$-calculus)~— вычислительный формализм
с тремя синтаксическим конструкциями, называемыми \emph{пре-лямбда-термами}:
\begin{itemize}
\item \emph{вхождение переменной}: $v$. При этом $v \in V$, где $V$~— некоторое множество имён переменных;
\item \emph{лямбда-абстракция}: $\lambda x.A$, где $x$~— имя переменной, а $A$~— пре-лямбда-терм. При этом терм $A$ называют \emph{телом абстракции}, а $x$ перед точкой~— \emph{связыванием}.
\item \emph{лямбда-аппликация}: $B C$;
\end{itemize}
и одной операцией \emph{бета-редукции}.
При этом говорят, что вхождение переменной является \emph{свободным},
если оно не связано какой-либо абстракцией.
Множество пре-лямбда-термов обозначают $\Lambda^{-}$.
\emph{Лямбда-термы}~— это пре-лямбда-термы, факторизованные по отношению \emph{альфа-эквивалентности}.
Обозначение: $\Lambda = \Lambda^{-} / =_{\alpha} $.

\emph{Альфа-эквивалентность} ($\alpha$-equality) отождествляет два пре-лямбда-терма, если один из них может быть получен из другого путём некоторого \emph{корректного} переименовывания переменных~— переименования не нарушающего отношение связанности.

\emph{Бета-редукция} ($\beta$-reduction) для лямбда-терма $A$ выбирает в нём некоторую лямбда-аппликацию $B C$, содержащую лямбда-абстракцию в левой части $A$, и заменяет свободные вхождения переменной, связанной $A$, в теле самой $A$ на терм $C$.\footnote{В терминах пре-лямбда-термов это означает замену свободных вхождений в теле $A$ на пре-терм $C$ так, чтобы ни для каких переменных не нарушилось отношение связанности. То есть, в пре-терме $A$ следует корректно переименовать все связанные переменные, имена которых совпадают с именами свободных переменных в $C$.}

Два лямбда-терма $A$ и $B$ называются \emph{конвертабельными},
когда существует две последовательности бета-редукций, приводящих их к общему терму $C$.
Или, эквивалентно, когда термы $A$ и $B$ состоят с друг с другом в рефлексивно-симметрично-транзитивном замыкании отношения бета-редукции, также называемом отношением \emph{бета-эквивалентности}.

За более подробной информацией об этом формализме следует обращаться к~\cite{TTFP} и~\cite{Sorensen}.

\section{Лямбда-исчисление с простыми типами}
\begin{definition}
    Пусть $U$ — бесконечное счетное множество, элементы которого мы будем
    называть \emph{переменными типов}.
    Множество \emph{простых типов} $\Pi$ — множество, определенное грамматикой:

    $$ \Pi ::= U \mid (\Pi \to \Pi) $$

    Для обозначения элементов множества $\Pi$ используют буквы греческого алфавита:
    $\sigma, \tau \ldots $.
\end{definition}
\begin{definition}
    Множество контекстов $C$ — это множество всех множеств пар такого вида:
    $$ \{ x_1 : \tau_1 , \ldots , x_n : \tau_n \} $$
    где $ \tau_1 , \ldots , \tau_n \in \Pi$, а
    $ x_1 , \ldots , x_n \in V $ (переменные из $\Lambda$) и $x_i \neq x_j$ если $i \neq j$.
\end{definition}
\begin{definition}
    \emph{Домен} контекста $\Gamma$ = $ \{ x_1 : \tau_1 , \ldots , x_n : \tau_n \} $:
    $$ \operatorname{dom} (\Gamma) = \{ x_1 , \ldots , x_n \} $$ и $x_i \neq x_j$ при $i \neq j$.
\end{definition}
\begin{definition}
    Отношение \emph{типизации} (typability) $ \vdash $ на множестве
    $ C \times \Lambda \times \Pi $ определяется следующими правилами:

    $$ \frac {} {\Gamma, x : \tau \vdash x : \tau}
    \quad
    \frac{\Gamma, x : \sigma \vdash M : \tau}
    {\Gamma \vdash \lambda x . M : \sigma \to \tau}
    \quad
    \frac{\Gamma \vdash M : \sigma \to \tau \quad \Gamma \vdash N : \sigma}
    {\Gamma \vdash M N : \tau}
    $$
    В первом и втором правиле мы требуем $x \notin \operatorname{dom}(\Gamma).$
\end{definition}
\begin{definition}
    \emph{Лямбда-исчисление с простыми типами} или $\lambda^{\to}$ — это тройка
    $ (\Lambda , \Pi , \vdash) $.
    Чтобы отличать данное в этой работе определение системы $\lambda^{\to}$ от других вариантов,
    эту систему называют лямбда-исчисление с простыми типами \emph{по Карри}.
\end{definition}
За более подробной информацией об этом формализме следует
обращаться к~\cite{ChurchSTLC} и~\cite{Sorensen}.

\section{Алгебраические типы данных и сопоставление с образцом}

\emph{Алгебраический тип данных} — вид составного типа, то есть типа,
сформированного комбинированием других типов.
Комбинирование осуществляется с помощью алгебраических операций — сложения и умножения.

\emph{Сумма} типов $A$ и $B$ — дизъюнктное объединение исходных типов.
Значения типа-суммы обычно создаются с помощью \emph{конструкторов}.

\emph{Произведение} типов $A$ и $B$ — прямое произведение исходных типов,
кортеж типов.

\subsection{Рекурсивные типы данных}
\emph{Рекурсивный тип данных} — тип данных, в определении которого содержится
определяемый тип данных. Например, список элементов типа $A$:
$$ List\;A = Nil + (A \times List\;A) $$

В теории~\cite{TAPL} для введения рекурсивных типов используются $\mu$-типы.
\emph{Сырые} $\mu$-типы вводятся с помощью оператора $\mu$: $\mu X . T $.
При этом $T$ может содержать $X$.

\begin{definition}
Сырой $\mu$-тип $T$ называется \emph{сократимым} (contractive),
если для любого подвыражения $T$ вида $ \mu X. \mu X_1 \ldots \mu X_n . S $
тело $S$ не равняется $X$.

Сырой $\mu$-тип называется просто \emph{$\mu$-типом} ($\mu$ -type), если он сократим.
\end{definition}
Пример: список элементов типа $A$: $ List\;A = \mu X . Nil + (A \times X)$.

\subsection{Сопоставление с образцом}

\emph{Сопоставление с образцом} — способ обработки
объектов % | значений |
алгебраических типов данных, который идентифицирует значения по конструктору
и извлекает данные в соответствии с представленным образцом.

\section{Теория типов}

\emph{Теория типов} — раздел математики изучающий отношения типизации вида
$ M \colon \tau $ и их свойства. $M$ называется \emph{термом} или \emph{выражением},
а $\tau$ — типом терма $M$.

Теория типов также изучает правила для \emph{переписывания} термов — замены
подтермов в выражениях другими термами.
Такие правила также называют правилами \emph{редукции} или \emph{конверсии} термов.
Редукцию терма $x$ в терм $y$ записывают: $x \to y$.
Также рассматривают транзитивное замыкание отношения редукции: $ \xrightarrow{*} $.
Например, термы $2 + 1$ и $3$ — разные термы, но первый редуцируется во второй:
$2 + 1 \xrightarrow{*} 3$.
Если для терма $x$ не существует терма $y$, для которого $x \to y$,
то говорят, что терм $x$ — в \emph{нормальной форме}.

\subsection{Отношение конвертабельности}

Два терма $x$ и $y$ называются \emph{конвертабельными},
если существует терм $z$ такой, что $x \xrightarrow{*} z$ и $y \xrightarrow{*} z$. Обозначают  $x \xleftrightarrow{*} y$.
Например, $1+2$ и $2+1$ — конвертабельны, как и термы
$x + (1 + 1)$ и $x + 2$. Однако, $x+1$ и $1+x$ (где $x$ — свободная переменная)
— не конвертабельны, так как оба представлены в нормальной форме.
Конвертабельность — рефлексивно-транзитивно-симметричное замыкание отношения
редукции.

\subsection{Интуиционистская теория типов}

Интуиционистская теория типов (теория типов Мартина-Лёфа)
основана на математическом конструктивизме~\cite{MLTT}.

Операторы для типов в ИТТ:
\begin{itemize}
    \item $\Pi$-тип (пи-тип) — зависимое произведение, обобщение типов функций ($ X \to Y $),
        в которых тип результата зависит от значения аргумента: $\Pi_{x : X} Y(x)$.
        Например, если $\operatorname{Vec}(A, n)$ — тип кортежей из $n$ элементов типа $A$,
        $\mathbb N$ — тип натуральных чисел, то
        $\Pi_{n \mathbin{:} {\mathbb N}} \operatorname{Vec}(A, n)$ 
        — тип функции, которая по натуральному числу $n$ возвращает кортеж из
        $n$ элементов типа $A$.
    \item $\Sigma$-тип — зависимая пара $\Sigma_{x : A} B(x)$.
        Второй элемент в зависимой паре зависит от первого.
        Например, тип $\Sigma_{n \mathbin{:} {\mathbb N}} \operatorname{Vec}(A, n)$ — тип 
        пары из числа $n$ и кортежа из $n$ элементов типа $A$.
    \item Пусть $A$ — множество конструкторов, $B$ — селектор на $A$.
        Элементы множества $A$ представляют разные способы сформировать
элемент в $W_{a : A} B(a) $, а $B(a)$ представляют части дерева, сформированные с помощью $a$.
        $W_{a : A} B(a) $  — рекурсивный тип, построенный с помощью конструкторов $B(a)$,
        который можно представить в виде \emph{фундированных деревьев} (well-founded trees)~\cite{WTypes}.
\end{itemize}
Базовые типы в ИТТ:
$\bot$ или $0$ — пустой тип, не содержащий ни одного элемента;
$\top$ или $1$ — единичный тип, содержащий единственный элемент.

\section{Унификация}

\emph{Унификатор} для термов $A$ и $B$ — подстановка $S$, действующая на их
свободные переменные, такая что $S(A) \equiv S(B)$.

\emph{Унификация} — процесс поиска унификатора.

\section{Agda}
\emph{Agda}~\cite{AgdaLang}~—  чистый функциональный язык программирования с зависимыми типами.
В Agda есть поддержка модулей:
\begin{code}\>\<%
\\
\>\AgdaKeyword{module} \AgdaModule{AgdaDescription} \AgdaKeyword{where}\<%
\\
\>\<\end{code} В коде на Agda широко используются символы Unicode.
Тип натуральных чисел — \D{ℕ}.
\begin{code}\>\<%
\\
\>\AgdaKeyword{data} \AgdaDatatype{ℕ} \AgdaSymbol{:} \AgdaPrimitiveType{Set} \AgdaKeyword{where}\<%
\\
\>[0]\AgdaIndent{2}{}\<[2]%
\>[2]\AgdaInductiveConstructor{zero} \AgdaSymbol{:} \AgdaDatatype{ℕ}\<%
\\
\>[0]\AgdaIndent{2}{}\<[2]%
\>[2]\AgdaInductiveConstructor{succ} \AgdaSymbol{:} \AgdaDatatype{ℕ} \AgdaSymbol{→} \AgdaDatatype{ℕ}\<%
\\
\>\<\end{code} \AgdaHide{
\begin{code}\>\<%
\\
\>\AgdaSymbol{\{-\#} \AgdaKeyword{BUILTIN} NATURAL \AgdaDatatype{ℕ} \AgdaSymbol{\#-\}}\<%
\\
\>\AgdaSymbol{\{-\#} \AgdaKeyword{BUILTIN} ZERO \AgdaInductiveConstructor{zero} \AgdaSymbol{\#-\}}\<%
\\
\>\AgdaSymbol{\{-\#} \AgdaKeyword{BUILTIN} SUC \AgdaInductiveConstructor{succ} \AgdaSymbol{\#-\}}\<%
\\
\>\<\end{code}
}
В Agda функции можно определять как mixfix операторы.
Пример — сложение натуральных чисел:
\begin{code}\>\<%
\\
\>\AgdaFunction{\_+\_} \AgdaSymbol{:} \AgdaDatatype{ℕ} \AgdaSymbol{→} \AgdaDatatype{ℕ} \AgdaSymbol{→} \AgdaDatatype{ℕ}\<%
\\
\>\AgdaInductiveConstructor{zero} \AgdaFunction{+} \AgdaBound{b} \AgdaSymbol{=} \AgdaBound{b}\<%
\\
\>\AgdaInductiveConstructor{succ} \AgdaBound{a} \AgdaFunction{+} \AgdaBound{b} \AgdaSymbol{=} \AgdaInductiveConstructor{succ} \AgdaSymbol{(}\AgdaBound{a} \AgdaFunction{+} \AgdaBound{b}\AgdaSymbol{)}\<%
\\
\>\<\end{code}
Символы подчеркивания обозначают места для аргументов.
% Система типов \textit{Agda} позволяет ... 
% В отличие от \textit{Haskell}, в \textit{Agda} имеется ... 

Зависимые типы позволяют определять типы, зависящие (индексированные) от значений
других типов. Пример — список, индексированный своей длиной:
\begin{code}\>\<%
\\
\>\AgdaKeyword{data} \AgdaDatatype{Vec} \AgdaSymbol{(}\AgdaBound{A} \AgdaSymbol{:} \AgdaPrimitiveType{Set}\AgdaSymbol{)} \AgdaSymbol{:} \AgdaDatatype{ℕ} \AgdaSymbol{→} \AgdaPrimitiveType{Set} \AgdaKeyword{where}\<%
\\
\>[0]\AgdaIndent{2}{}\<[2]%
\>[2]\AgdaInductiveConstructor{nil} \<[7]%
\>[7]\AgdaSymbol{:} \AgdaDatatype{Vec} \AgdaBound{A} \AgdaInductiveConstructor{zero}\<%
\\
\>[0]\AgdaIndent{2}{}\<[2]%
\>[2]\AgdaInductiveConstructor{cons} \AgdaSymbol{:} \AgdaSymbol{∀} \AgdaSymbol{\{}\AgdaBound{n}\AgdaSymbol{\}} \AgdaSymbol{→} \AgdaBound{A} \AgdaSymbol{→} \AgdaDatatype{Vec} \AgdaBound{A} \AgdaBound{n} \AgdaSymbol{→} \AgdaDatatype{Vec} \AgdaBound{A} \AgdaSymbol{(}\AgdaInductiveConstructor{succ} \AgdaBound{n}\AgdaSymbol{)}\<%
\\
\>\<\end{code}
В фигурные скобки заключаются неявные аргументы.

Такое определение позволяет нам описать функцию $ \F{head} $ для такого списка, которая не может бросить исключение:
\begin{code}\>\<%
\\
\>\AgdaFunction{head} \AgdaSymbol{:} \AgdaSymbol{∀} \AgdaSymbol{\{}\AgdaBound{A}\AgdaSymbol{\}} \AgdaSymbol{\{}\AgdaBound{n}\AgdaSymbol{\}} \AgdaSymbol{→} \AgdaDatatype{Vec} \AgdaBound{A} \AgdaSymbol{(}\AgdaInductiveConstructor{succ} \AgdaBound{n}\AgdaSymbol{)} \AgdaSymbol{→} \AgdaBound{A}\<%
\\
\>\<\end{code}
У аргумента функции $ \F{head} $ тип $ \D{Vec}\,A\,(\DC{succ}\,n) $, то есть вектор, в котором есть хотя бы один элемент.
Это позволяет произвести сопоставление с образцом только по конструктору $ \DC{cons} $:
\begin{code}\>\<%
\\
\>\AgdaFunction{head} \AgdaSymbol{(}\AgdaInductiveConstructor{cons} \AgdaBound{a} \AgdaBound{as}\AgdaSymbol{)} \AgdaSymbol{=} \AgdaBound{a}\<%
\\
\>\<\end{code}



\section{Индуктивные семейства}

\begin{definition}
\emph{Индуктивное семейство}~\cite{DybjerIndFam, RefiningIT}~— это индуктивный тип данных,
который может зависеть от других типов и значений.
Тип или значение, от которого зависит зависимый тип, называют \emph{индексом}.
\end{definition}

Одной из областей применения индуктивных семейств являются системы интерактивного
доказательства теорем.

Индуктивные семейства позволяют формализовать математические структуры,
кодируя утверждения о структурах в них самих,
тем самым перенося сложность из доказательств в определения.

\section{Использование индуктивных семейств в структурах данных}
В работах~\cite{HongweiXi, McBridePivotal} приведены различные подходы
в использовании индуктивных семейств в реализации структур данных
и доказательств их свойств.

Пример задания структуры данных и инвариантов — тип данных AVL-дерева
и тип данных для хранения баланса высоты поддеревьев в AVL-дереве~\cite{AVLTree}.

\AgdaHide{
\begin{code}\>\<%
\\
\>\AgdaKeyword{module} \AgdaModule{AVLBalance} \AgdaKeyword{where}\<%
\\
\>[0]\AgdaIndent{2}{}\<[2]%
\>[2]\AgdaKeyword{open} \AgdaKeyword{import} \AgdaModule{AgdaDescription}\<%
\\
\>[0]\AgdaIndent{2}{}\<[2]%
\>[2]\AgdaKeyword{infix} \AgdaNumber{4} \_∼\_\<%
\\
\>[0]\AgdaIndent{2}{}\<[2]%
\>[2]\AgdaFunction{testˡ} \AgdaSymbol{:} \AgdaDatatype{ℕ}\<%
\\
\>[0]\AgdaIndent{2}{}\<[2]%
\>[2]\AgdaFunction{testˡ} \AgdaSymbol{=} \AgdaInductiveConstructor{succ} \AgdaInductiveConstructor{zero}\<%
\\
\>\<\end{code}
}
Если $m \sim n$, то разница между $m$ и $n$ не больше чем один:
\begin{code}\>\<%
\\
\>[0]\AgdaIndent{2}{}\<[2]%
\>[2]\AgdaKeyword{data} \AgdaDatatype{\_∼\_} \AgdaSymbol{:} \AgdaDatatype{ℕ} \AgdaSymbol{→} \AgdaDatatype{ℕ} \AgdaSymbol{→} \AgdaPrimitiveType{Set} \AgdaKeyword{where}\<%
\\
\>[2]\AgdaIndent{4}{}\<[4]%
\>[4]\AgdaInductiveConstructor{∼+} \AgdaSymbol{:} \AgdaSymbol{∀} \AgdaSymbol{\{}\AgdaBound{n}\AgdaSymbol{\}} \AgdaSymbol{→} \<[21]%
\>[21]\AgdaBound{n} \AgdaDatatype{∼} \AgdaNumber{1} \AgdaFunction{+} \AgdaBound{n}\<%
\\
\>[2]\AgdaIndent{4}{}\<[4]%
\>[4]\AgdaInductiveConstructor{∼0} \AgdaSymbol{:} \AgdaSymbol{∀} \AgdaSymbol{\{}\AgdaBound{n}\AgdaSymbol{\}} \AgdaSymbol{→} \<[21]%
\>[21]\AgdaBound{n} \AgdaDatatype{∼} \AgdaBound{n}\<%
\\
\>[2]\AgdaIndent{4}{}\<[4]%
\>[4]\AgdaInductiveConstructor{∼-} \AgdaSymbol{:} \AgdaSymbol{∀} \AgdaSymbol{\{}\AgdaBound{n}\AgdaSymbol{\}} \AgdaSymbol{→} \AgdaNumber{1} \AgdaFunction{+} \AgdaBound{n} \AgdaDatatype{∼} \AgdaBound{n}\<%
\\
\>\<\end{code}


В работе~\cite{McBridePivotal} представлен способ обобщения
упорядоченных структур данных
(таких как отсортированные списки и деревья поиска)
и использование этого метода для реализации 2-3 деревьев.

\section{Выводы по главе~\ref{chapter1}}

В этой главе были рассмотрены некоторые существующие подходы к построению структур данных
с использованием индуктивных семейств.
Кратко описаны особенности языка программирования \textit{Agda},
который использован в этой работе.

% \chapter{Описание реализованного подхода}
\label{chapter2}

В данной главе описывается разработанный метод автоматизированного покрытия кода тестами на основе эволюционных алгоритмов.

\section{Постановка задачи}
Целью данной работы является создание платформы для автоматизированного покрытия программ тестами, учитывающими внутреннюю структуру тестируемой программы, на 
основе эволюционных алгоритмов и проверка разработанного метода на ряде модельных задач. Требования к данной работе:
\begin{itemize}
 \item Разработать метод автоматизированного покрытия тестами кода программ, работающих на \textit{JVM}.
 \item Провести сравнение разработанного метода с методом генерации случайных тестов на ряде модельных задач.
 \item Апробировать предложенный подход для покрытия тестами решения олимпиадной задачи Huzita~Axiom~6. 
\end{itemize}

\section{Общая схема решения}


На рисунке~\ref{pipeline} показана общая схема решения. При загрузке \textit{class}-файла тестируемой программы происходит модификация кода с целью получения 
траектории выполнения, а также считывания значений, переданных инструкциям ветвления, в процессе выполнения модифицированной программы. Заметим, что модификация 
кода тестируемой программы происходит таким образом, чтобы не изменился результат ее работы.

\begin{figure}[h!]
  \center{\includegraphics[width=0.6\textwidth]{pipeline.pdf}}
  \caption{Общая схема алгоритма}
  \label{pipeline}
\end{figure} 

После загрузки модифицированного кода происходит инициализация списка ветвлений, покрываемых инструкций. Будем называть целью заданное ветвление выбранной 
инструкции. Дальнейшее решение сводится к поиску набора тестов, обеспечивающих покрытие каждой цели.

Построение теста, покрывающего заданную цель, рассматривается как задачу оптимизации, решаемая с помощью эволюционных алгоритмов (ЭА). В качестве особи 
ЭА кодируется тест. С учетом требований к искомому тесту для особи определяются необходимые эволюционные операции: мутация и кроссовер. 

Для покрытия каждой цели используется свой экземпляр ЭА. Экземпляры ЭА создаются лишь для непокрытых целей. Если в процессе работы находится тест, 
покрывающий ранее непокрытую цель, то такой тест сохраняется.

После получения набора тестов, покрывающего выбранные цели, производится его минимизация с целью уменьшения накладных расходов при дальнейшем тестировании.

Таким образом, для запуска алгоритма необходимо:
\begin{itemize}
 \item задать список целей;
 \item сконфигурировать ЭА;
 \item обеспечить возможность многочисленных расчетов функции приспособленности.
\end{itemize}

\section{Функция приспособленности}
\label{sec:fitness_fun}
Функция приспособленности (ФП) дает количественную оценку того, насколько заданный тест покрывает выбранную цель. В данной работе ФП выбирается так, чтобы 
оценивать, насколько близка траектория выполнения программы к заданной цели.

\subsection{Функция расстояния до ветви}
\label{sec:branch_distance}
У некоторых инструкций ветвления лишь определенные ветви способствуют покрытию тестами выбранной цели, поскольку только они приводят к выполнению нужного 
фрагмента кода. Функция расстояния до ветви определяется таким образом, что ее минимальное значение достигается при прохождении по желаемому ветвлению.

Рассмотрим построение функции расстояния до нужной ветви на примере метода, код которого приведен на листинге~\ref{lst:branch_distance}.

\begin{snippet}[caption=Пример функции расстояния до ветви, label={lst:branch_distance}]
  def testMethod(x : Int) {
    if (10 <= x && x <= 15) {
      A()	// %*целевое ветвление*)
    }
    B()
  }
\end{snippet}

На рисунке~\ref{branch_distance} представлена блок-схема тестируемого метода. Заметим, что условие $10 \le x \le 15$ заменяется двумя инструкциями ветвления на 
этапе компиляции. В качестве функции расстояния до ветви в данном случае подойдут $10 - x$ и $x - 15$ для первой и второй инструкции ветвления соответственно. 

\begin{figure}[h!]
  \center{\includegraphics[width=0.5\textwidth]{branch_distance.pdf}}
  \caption{Блок-схема для листинга~\ref{lst:branch_distance}. Малиновым цветом выделена целевая траектория выполнения.}
  \label{branch_distance}
\end{figure}

Качество теста оценивается в зависимости от выбора функции расстояния до ветви. Так, например, сравнение значений типов long, double и float происходит при 
помощи специализированных инструкций, как \texttt{lcmp},  \texttt{dcmpg}, \texttt{dcmpl}, \texttt{fcmpg}, \texttt{fcmpl}. Результатом выполнения этих 
инструкций будет ноль, единица или минус единица и ветвление потока управления происходит с использованием соответствующих инструкций сравнения с нулем. Если в 
качестве значения функции расстояния до ветви использовать значения, переданные в инструкции ветвления напрямую, то эволюционный алгоритм будет работать плохо. 
Вместо них лучше использовать значения, переданные непосредственно в специализированные инструкции сравнения. Поэтому при подсчете функции расстояния до ветви 
необходимо анализировать код тестируемой программы.

\subsection{Следование опорной траектории}
\label{sec:track_based_fitness}
В данном случае предполагается, что траектория выполнения тестируемой программы должна совпадать с выбранной опорной траекторией. Опорная траектория либо 
задается человеком, либо генерируется автоматически на основе графа потока управления.

Рассмотрим опорную траекторию, состоящую из целей $C_1,C_2,..,C_k$. Пусть на некотором тесте была получена траектория $C_1,C_2,..,C_t,J,..$, где $J \neq 
C_{t+1}$. ФП представляется в виде пары двух чисел. Первое~--- длина общего префикса двух траекторий, а именно \texttt{t}. В качестве второго числа берется 
значение функции расстояния до ветви для цели $C_t$, поскольку именно в ней разошлись две траектории. При сравнении двух особей более приспособленной считается 
та, у которой первое число ФП больше. Если первые числа ФП равны, то лучше та особь, у которой второе число меньше.

Рассмотрим пример на рис.~\ref{singlepathfitness}. Опорная траектория выделена малиновым цветом. Зелеными кружками помечена траектория выполнения программы. 
Инструкция, отмеченная темно-зеленым кружком, соответствует $C_t$ и в ней происходит расхождение траекторий. Длина общего префикса отмеченных траекторий 
равна трем.

\begin{figure}[h!]
  \center{\includegraphics[width=0.5\textwidth]{singlepathfitness.pdf}}
  \caption{Пример расхождения с опорной траекторией}
  \label{singlepathfitness}
\end{figure}

Существуют другие варианты построения функции приспособленности на основе следования опорной траектории. Допустим, ни одна траектория выполнения программы не 
может следовать выбранной опорной траектории во всех инструкциях. В таком случае невозможно найти тест, покрывающий выбранную цель. Однако если в 
качестве первого числа функции приспособленности взять наибольший номер инструкции, общей для двух траекторий, то такой тест может быть сгенерирован.

Недостатком такого подхода является необходимость выбора опорной траектории, поскольку возможно большое число траекторий, проходящих по заданному ветвлению. 
При этом время работы алгоритма сильно зависит от выбора опорной траектории, так как не вдоль всех траекторий можно сгенерировать тест.

\subsection{Приближение по многим направлениям}
Обычно при покрытии выбранной цели нет предпочтений насчет траектории, вдоль которой надо покрывать. Таким образом, необходимо выбрать опорную траекторию, либо 
задать функцию приспособленности, рассматривающую приближение вдоль всех возможных траекторий.

Расстоянием \texttt{d(x, y)} между инструкцией \texttt{x} и инструкцией \texttt{y} будем считать минимальную длину пути от инструкции \texttt{x} до инструкции 
\texttt{y} в графе потока управления. Если такой путь не существует, то $d(x, y) = \infty$. В качестве расстояния \texttt{d(T, x)} от траектории $T={t_1, 
t_2,...,t_k}$ до инструкции \texttt{x} возьмем $\min\limits_{i=1..k}{d(t_i, x)}$.


Пусть выбрана цель \texttt{c} и заданному тесту \texttt{t} соответствует траектория выполнения \texttt{T}. Тогда в качестве значения функции приспособленности 
берется пара чисел, первое из которых равно \texttt{d(T, c)}, а второе~--- минимальному значению функции расстояния до ветви для инструкции \texttt{i} на 
траектории \texttt{T}, такой что $d(i,c) = d(T,c)$. 

Рассмотрим пример на рис.~\ref{multipathfitness}. На нем изображены две различные траектории выполнения одного и того же фрагмента кода, выделенные зеленым 
цветом. Цель \texttt{c} выделена малиновым цветом. Темно-зеленым цветом выделена инструкция \texttt{i}. Обе траектории находятся на расстоянии до цели, равном 
одному условному переходу.

\begin{figure}[h!]
  \center{\includegraphics[width=0.8\textwidth]{multipathfitness.pdf}}
  \caption{Пример приближения по многим направлениям}
  \label{multipathfitness}
\end{figure}

В данном примере, при совпадении второго числа в функции приспособленности, количественные оценки для обеих траекторий будут одинаковые, но качественно эти 
траектории различаются. Таким образом, при данном определении функции приспособленности одинаковые количественные оценки могут соответствовать качественно 
различным тестам.          

\section{Модификация кода}

Модификации кода тестируемой программы не должны изменять результат ее работы. Для работы с \textit{class}-файлами используется библиотека 
ASM~4.0~\cite{asm_lib}, предоставляющая доступ к списку инструкций. Базовый интерфейс применяемых модификаций приведен на листинге~\ref{lst:modification}.

\begin{snippet}[caption=Типаж модификации кода тестируемой программы, label={lst:modification}]
  import org.objectweb.asm.tree.{ClassNode, MethodNode}

  trait CodeModification {
    protected def onMethod(cn : ClassNode, mn: MethodNode)
  
    def onClass(cn : ClassNode) {
      cn.methods.iterator.foreach{ x =>
	onMethod(cn, x.asInstanceOf[MethodNode])
      }
    }
  }
\end{snippet}

Для модификации кода метода необходимо переопределить метод \texttt{onMethod}, входными аргументами которого являются ссылки на инструкции модифицируемого 
метода и класса, в котором этот метод определен. Для композиция двух модификаций метод \texttt{onMethod} определяется как последовательность вызовов методов 
\texttt{onMethod} этих модификаций. Класс композиции представлен на листинге~\ref{lst:composition}.

\begin{snippet}[caption=Композиция модификаций кода тестируемой программы, label={lst:composition}]
  import org.objectweb.asm.tree.{ClassNode, MethodNode}

  class Composition(a : CodeModification, b : CodeModification) extends CodeModification {
    protected def onMethod(cn : ClassNode, mn: MethodNode) {
      a.onMethod(cn, mn)
      b.onMethod(cn, mn)
    }
  }
\end{snippet}

Работа с кодом программы разделена на два этапа: на первом формируется внутреннее представление программы в виде графа потока управления, на втором~--- 
код модифицируется с целью считывания траектории выполнения программы и другой информации о процессе выполнения.

\subsection{Граф потока управления}
Библиотека \texttt{ASM} предоставляет доступ к списку инструкций каждого метода. Однако для структурного тестирования необходимо построение графа потока 
управления. 

\begin{definition}
Графом потока управления программы \texttt{F} называется ориентированный граф $G=(N, E, s, F)$, где \texttt{N}~--- множество вершин, \texttt{E}~--- множество 
ребер, \texttt{s}~--- входная инструкция, \texttt{F}~--- множество вершин, завершающих выполнение программы.
\end{definition}

Каждой вершине $n \in N$ графа потока управления соответствует инструкция \textit{JVM}. Ребром $e = (n_i, n_j)$ же является переход от инструкции $n_i$ к 
инструкции $n_j$.

Для построения графа потока управления используется типаж анализатора кода, представленный на листинге~\ref{lst:asm_analyzer}. При помощи дополнительных 
типажей анализатор индексирует инструкции ветвления, а также сохраняет построенный граф потока управления.

\begin{snippet}[caption=Типаж анализатора кода, label={lst:asm_analyzer}]
  import org.objectweb.asm.tree.{ClassNode, MethodNode}
  
  trait AsmControlFlowAnalyzer extends CodeModification { 
    needs : InstructionRegister with MethodRegister =>

    protected override def onMethod(cn : ClassNode, mn: MethodNode) {
      ...
    }
  }
\end{snippet}

\subsection{Считывание траектории выполнения программы}
Для анализа выполнения программы на заданном тесте считывается траектория выполнения, а именно последовательность инструкций ветвления и значения переданные им 
во время выполнения. Для идентификации инструкций ветвления используется нумерация, введенная при построении графа потока управления. 

Для считывания значений во время выполнения программы определен ряд методов, каждый из которых соответствует определенному типу инструкции ветвления. При работе 
с несколькими потоками последовательность инструкций ветвления будет составлена для каждого потока в отдельности~\cite{threadlocal}.

Рассмотрим модификацию кода программы, необходимую для считывания траектории выполнения. Для каждого профилируемого метода вводятся дополнительные локальные 
переменные. Перед выполнением инструкции ветвления выполняются следующие действия:
\begin{enumerate}
 \item Значения со стека операндов сохраняются в дополнительные локальные переменные.
 \item Состояние стека операндов восстанавливается из локальных переменных.
 \item Вызывается метод для считывания значений во время выполнения.
 \item Восстанавливается состояние стека операндов для дальнейшей работы программы.
\end{enumerate}

\section{Минимизация набора тестов}

Минимизация набора тестов является частным случаем задачи о минимальном покрытии множества. Пусть имеется множество тестов $T$. Для каждого теста $t \in T$ 
известно множество фрагментов кода, которые он покрывает. Требуется построить минимальное по мощности множество $T_{opt} \subset T$, такое что: 
$\bigcup\limits_{t \in T_{opt}}Cov(t) = \bigcup\limits_{t \in T}Cov(t)$.

В работе \cite{minimization} показано, что данная задача является \textit{NP}-полной. По причине чрезмерных затрат вычислительных ресурсов 
требуется применять приближенные методы.

Рассмотрим жадный алгоритм, решающий данную задачу:
\begin{enumerate}
 \item Пусть множество $X$~--- множество выбранных тестов, $P = \bigcup\limits_{t \in T}Cov(t)$, а множество $Z$~--- множество покрытых им фрагментов кода. 
Изначально $X, Z = \emptyset$.
 \item Если $Z = P$, прекратить работу, решение задачи $T_{opt} = X$.
 \item Выбрать такое $t \in T$, что $Cov(t) \cap (P / Z)$ максимально.
 \item $X = X \cup {t}$, $Z = Z \cup Cov(t)$, перейти к шагу 2.
\end{enumerate}

Время работы простейшая реализация данного алгоритма имеет асимптотическую оценку $O(|T|^2 \cdot |P|)$.  В работе~\cite{minimization_characterization} 
показано, что данный алгоритм генерирует в худшем случае ответ, превосходящий оптимальный в $O(log|P|)$ раз. Однако на практике для многих входных данных он 
дает ответ, отличающийся от оптимального не более чем на 10\%.

\section{Выводы по главе \ref{chapter2}}
Формализована цель работы: создание платформы для автоматизированного покрытия программ тестами на основе эволюционных алгоритмов. Описан предлагаемый подход. 
Предложен способ задания функции приспособленности, рассматривающий приближение по многим направлениям.
% \chapter{Результаты} 
\label{chapter3}

В данном разделе описываются результаты экспериментов, демонстрирующих работу предложенного метода генерации покрывающего набора тестов. 

\section{Покрытие тестами модельных задач}
Исследования проводились на пяти модельных задачах, взятых с сайта инструментального средства Microsoft Pex \texttt{http://pexforfun.com}. 

\subsection{Описание модельных задач}
Для каждой задачи приводится исходный код на языке \textit{C\#}, описывается способ кодирования теста и оператор мутации. Для построения покрывающего набора 
тестов использовалась \texttt{(1+1)}-эволюционная стратегия.

В рассматриваемых задачах при применении оператора мутации для изменения целого числа \texttt{x} к нему добавлялось число вида $(r(19) - 9) \cdot 10 ^{r(10)}$, 
где $r(a)$~--- случайное целое число в диапазоне $[0, a)$. Если \texttt{x} должен находиться в диапазоне от $-10^5$ до $10^5$, то он заменяется на $\max(-10^5, 
\min(10^5, x + (r(19) - 9) \cdot 10^{r(4)}))$. 

\subsubsection{Задача 1}
\label{pex01}
Код тестируемой программы приведен на листинге~\ref{lst:pex-1-source}. Наиболее сложным для покрытия является случай, когда сумма элементов массива равняется 
42. В качестве тестов генерируются целочисленные массивы длины шесть. Оператор мутации выбирает случайный элемент массива и изменяет его описанным ранее 
способом.

\begin{snippet}[language=C++,caption={Код задачи 1 с сайта pexforfun},label={lst:pex-1-source}]
using System;

public class Program {
  //#What values of v to cause the exception? Ask Pex to find out!#
  public static int Puzzle(int[] v) {
    int sum = 0;
    foreach (int x in v)
      sum += x;
    if (sum == 42)
      throw new Exception("hidden bug!");
    return sum;
  }
}
\end{snippet}

\subsubsection{Задача 2}

Код тестируемой программы приведен на листинге~\ref{lst:pex-2-source}. Сложность данной задачи заключается в покрытии случая, когда элемент списка, следующий 
за указанным, на единицу больше. Тестом является список, состоящий из шести целых чисел, и число из множества $\{0, 1, 2, 3, 4\}$~--- в качестве индекса 
элемента в списке. Оператор мутации случайным образом выбирает другой индекс, либо изменяет элемент списка, выбранный случайным образом.  

\begin{snippet}[language=C++,caption={Код задачи 2 с сайта pexforfun},label={lst:pex-2-source}]
using System;
using System.Collections.Generic;

public class Program
{
  //# What values of list and i can cause exceptions? Ask Pex to find out!#
  public static void Puzzle(List<int> list, int i)
  {
    if (list[i] + 1 == list[i + 1])
      throw new Exception("hidden bug!"); 
  }
}
\end{snippet}

\subsubsection{Задача 3}
                   
Код тестируемой программы приведен на листинге~\ref{lst:pex-3-source}. Сложнее всего покрыть случай, когда сумма выбранного элемента массива и 27277 равняется 
42. Тест представляется в виде массива из шести целых чисел и числа из множества $\{0, 1, 2, 3, 4, 5\}$~--- индекса элемента в массиве. Оператор мутации 
случайным образом выбирает другой индекс, либо изменяет элемент массива, выбранный случайным образом.  

\begin{snippet}[language=C++,caption={Код задачи 3 с сайта pexforfun},label={lst:pex-3-source}]
using System;

public class Program {
  //# What values of v and i can cause an exception? Ask Pex to find out!#
  public static void Puzzle(int[] v, int i) {
    if (v[i] + 27277 == 42)
      throw new Exception("hidden bug!"); 
  }
}
\end{snippet}

\subsubsection{Задача 4}
                   
Код тестируемой программы приведен на листинге~\ref{lst:pex-4-source}. Чтобы полностью покрыть данную задачу, необходимо подобрать корень линейного уравнения. 
В качестве теста берется целое число в диапазоне от $-10^5$ до $10^5$. Оператор мутации изменяет текущее решение описанным ранее образом.

\begin{snippet}[language=C++,caption={Код задачи 4 с сайта pexforfun},label={lst:pex-4-source}]
using System;

public class Program 
{
  public static void Puzzle(int x)
  {
    // What value of x solves this equation? Ask Pex to find out!
    if (x * 3 + 27 == 153)
      Console.WriteLine("equation solved");
  }
}
\end{snippet}

\subsubsection{Задача 5}
                                                            
Код тестируемой программы приведен на листинге~\ref{lst:pex-5-source}. Сложнее всего покрыть тестом строку, в которой производится вывод на консоль.
Это эквивалентно решению нелинейного уравнения второй степени в целых числах с двумя переменными и наличием ограничений. Тест представляется в виде кортежа из 
двух целых чисел в диапазоне от $-10^5$ до $10^5$. Оператор мутации изменяет одно из них.

\begin{snippet}[language=C++,caption={Код задачи 5 с сайта pexforfun},label={lst:pex-5-source}]
using System;

public class Program {
  public static void Puzzle(int x, int y) {
    //# What values of x and y solve this equation? Ask Pex to find out!#
    if (x >= 0 && x <= 100 &&
        y >= 0 && y <= 100 &&
        x * y - 37 * y + 71 * x - 2627 == 0)        
      Console.WriteLine("equation solved"); 
  }
}
\end{snippet}

\subsection{Результаты эксперимента}

Для каждой из рассмотренных задач было проведено 1000 запусков эволюционного алгоритма. В результате каждого из запусков был сгенерирован покрывающий набор 
тестов. В таблице~\ref{pexres} приведены минимальное, среднее и максимальное число вычислений функции приспособленности (ФП).

\begin{table}[h!]
\caption{Число вычислений ФП на модельных задачах} \label{pexres}
\begin{tabular}{|p{5em}|c|c|c|}
\hline
№ задачи & ФП, мин. & ФП, среднее & ФП, макс. \\\hline
%Pex #01
1  & 33  &  340 & 1449 \\\hline
%Pex #02
2 & 2  &  660& 3909 \\\hline
%Pex #03
3  & 286  &  2604 & 13078 \\\hline
%Pex #04
4  & 20  &  201 & 679 \\\hline
%Pex #05
5  & 298  &  1429 & 4086 \\\hline
\end{tabular}
\end{table}

Покрытие модельных задач было осуществлено с помощью тестов, сгенерированных случайным образом. При этом, чтобы было возможным полное покрытие заданного кода, 
пространство поиска тестов было уменьшено. Так, например, для задачи \ref{pex01} массив заполнялся целыми числами в диапазоне от $-5 \cdot 10^5$ до $5 \cdot 
10^5$. При этом на каждом запуске генерировалось в $10$ раз больше тестов, чем максимальное число вызовов ФП для соответствующей задачи. В таблице 
\ref{randompex} приведены полученные результаты, усредненные по 1000 запускам.

\begin{table}
\caption{Результаты покрытия модельных задач случайными тестами} \label{randompex}
\begin{tabular}{|p{5em}|c|c|}
\hline
№ задачи & Покрытие, \% & Число тестов на каждом запуске\\\hline
% Pex #01
1 & 75.3 & 15000\\\hline
% Pex #02
2 & 52.7 & 40000\\\hline
% Pex #03
3 & 55.6 & 130000\\\hline
% Pex #04
4 & 51.4 & 7000\\\hline
% Pex #05
5 & 66.7 & 41000\\\hline
\end{tabular}
\end{table}

Из результатов эксперимента можно сделать вывод, что применение эволюционного алгоритма для генерации тестов, покрывающих заданные строки кода, весьма 
эффективно даже для сложных условий.

\section{Покрытие тестами олимпиадных задач}
Исследования проводились на основе задачи  Huzita Axiom 6, предложенной на NEERC 2011. 

\subsection{Условие задачи}
Заданы две прямые $l_1$ и $l_2$ и две точки $p_1$ и $p_2$. Необходимо найти прямую, по которой можно сложить плоскость так, что точка $p_1$ попадет на прямую 
$l_1$, а точка $p_2$ попадет на прямую $l_2$. Прямые задаются с помощью двух точек. Пример условия задачи приведен на рис.~\ref{huzita_pic}.
\begin{figure}[h!]
 \center{\includegraphics[width=0.8\textwidth]{huzita.pdf}}
 \caption{Иллюстрация к задаче  Huzita Axiom 6}
 \label{huzita_pic}
\end{figure}

\subsection{Описание эксперимента}

Для тестирования решений задачи был разработан интерфейс \textit{Solution}, приведенный на листинге~\ref{lst:huzita_solution}. Решения реализуют данный 
интерфейс. Входные данные(тест) передаются в виде строки, а результат выдается в виде списка строк.

\begin{snippet}[language=Java,caption={Интерфейс решения задачи Huzita Axiom 6},label={lst:huzita_solution}]
public interface Solution {
    public List<String> solve(String input) throws IOException;
}
\end{snippet}

Для построения покрывающего набора тестов использовалась \texttt{(1+1)}-эволюционная стратегия. Пример конфигурации ЭА, приведен на 
листинге~\ref{lst:huzita_config}.

\begin{snippet}[language=Scala,caption={Конфигурация эволюционного алгоритма для задачи  Huzita Axiom 6},label={lst:huzita_config}]
class HuzitaConfig[C <: Solution](val cl : Class[C], override val maxGenerationCount : Long) 
  extends  TargetAwareComputationStateImpl
  with CoverageConfig[TestData, DistDiffFitness]
  with OnePlusOneES
  with TraceProvider
  with Mutation
  with FastRandomDelegate
  with CoverageFitnessComparator.LeastDistLeastAbsDiff[DistDiffFitness]
  with GenotypeGenerator
  with GenotypeWrapper
  with TotalFitnessCallCount
{
  private[this] def nestedClasses(top : Class[_]) : Seq[Class[_]] = {
    top.getDeclaredClasses.toSeq.map(x => nestedClasses(x)).flatten :+ top
  }

  override def targetMethods: Seq[Method] = nestedClasses(cl).map(_.getDeclaredMethods).flatten

  override def targetConstructors: Seq[Constructor[_]] = nestedClasses(cl).map(_.getDeclaredConstructors).flatten

  def newGenotype(): TestData = TestData.genTestData()

  private[this] lazy val mh =  MethodHandles.lookup().findVirtual(cl, "solve", MethodType.methodType(classOf[java.util.List[String]], classOf[String]))

  def wrap(genotype: TestData): Seq[AnyRef] = genotype.toTestString :: Nil

  def mutate(source: TestData): TestData = source.mutate()

  def evaluateFitness(genotype: TestData): DistDiffFitness = computeFitness(wrap(genotype))

  protected def invoke(args: Seq[AnyRef]) {
    mh.invokeWithArguments(cl.newInstance() +: args : _*)
  } 
} 
\end{snippet}

Точка задается двумя целочисленными координатами. При мутации точки изменяются обе ее координаты. Линия определяется с помощью двух точек, и при мутации 
изменяется каждая из них. Тест, используемый в качестве особи ЭА, задается с помощью двух прямых~--- $l_1$ и $l_2$~--- и двух точек~---$p_1$ и $p_2$. 
Рассматривается пять типов операторов мутации тестовых данных.
Мутируют:
\begin{enumerate}
 \item либо $l_1$ и $p_1$, либо $l_2$ и $p_2$;
 \item либо $l_1$ и $l_2$, либо $p_1$ и $p_2$;
 \item одна из $l_1$, $l_2$, $p_1$, $p_2$;
 \item все прямые и все точки;
 \item одна из $l_1$ и $p_1$ и одна из $l_2$ и $p_2$.
\end{enumerate}

\subsection{Результаты}

Для каждого эксперимента было проведено по 1000 запусков. В таблицах~\ref{huzita_10_res}, \ref{huzita_100_res} и \ref{huzita_1000_res} приведены результаты 
работы ЭА, когда координаты точек лежат в диапазоне от $-10$ до 10, от $-100$ до 100 и от $-1000$ до 1000 соответственно. В первой колонке указан номер 
оператора мутации, во второй~--- среднее число вычислений ФП, а в третьей~--- усредненный процент покрытия кода. Можно видеть, что лучшие результаты были 
получены при использовании пятого оператора мутации.

\begin{table}[h!]
\caption{ЭА, диапазон от $-10$ до 10} \label{huzita_10_res}
\begin{tabular}{|p{5em}|c|c|}
\hline
№ мутации &  ФП, среднее & Покрытие, \% \\\hline
1  & $2,5 \cdot 10^4$ &  95,8 \\\hline
2  & $2,9 \cdot 10^4$  &  95  \\\hline
3  & $4,5 \cdot 10^4$  &  89,3 \\\hline
4  & $2,5 \cdot 10^4$  &  95,9 \\\hline
5  & $2,2 \cdot 10^4$  &   \cellcolor{purpur}96,6 \\\hline
\end{tabular}
\end{table}

\begin{table}[h!]
\caption{ЭА, диапазон от $-100$ до 100} \label{huzita_100_res}
\begin{tabular}{|p{5em}|c|c|}
\hline
№ мутации &  ФП, среднее & Покрытие, \% \\\hline
1  & $5,7 \cdot 10^5$ & 87,3 \\\hline
2  & $4,7 \cdot 10^5$  &  89,7  \\\hline
3  & $5,7 \cdot 10^5$  &  87,3 \\\hline
4  & $6,7 \cdot 10^5$  &  88,3 \\\hline
5  & $4,2 \cdot 10^5$  &  \cellcolor{purpur}91,3 \\\hline
\end{tabular}
\end{table}

\begin{table}[h!]
\caption{ЭА, диапазон от $-1000$ до 1000} \label{huzita_1000_res}
\begin{tabular}{|p{5em}|c|c|}
\hline
№ мутации &  ФП, среднее & Покрытие, \% \\\hline
1  & $7,4 \cdot 10^5$  & 81,7 \\\hline
2  & $6,7 \cdot 10^5$  &  87,3  \\\hline
3  & $6,2 \cdot 10^5$ & 88 \\\hline
4  & $7,9 \cdot 10^5$  &  79,3 \\\hline
5  & $5,9 \cdot 10^5$ &  \cellcolor{purpur}90 \\\hline
\end{tabular}
\end{table}

В таблице~\ref{huzita_random_res} приведены результаты покрытия кода случайными тестами в сравнении с результатами ЭА. В первой колонке указан диапазон 
координат, во второй~--- число случайно генерируемых тестов, в третьей~--- усредненный процент покрытия кода случайными тестами, а в четвертой~--- лучший 
усредненный процент покрытия кода с помощью ЭА.

\begin{table}
\caption{Покрытие решения Huzita Axiom 6 случайными тестами} \label{huzita_random_res}
\begin{tabular}{|p{7em}|c|c|c|}
\hline
Диапазон &  Число тестов & Покрытие, \% & Покрытие ЭА, \%\\\hline
$-10$ до 10  & $5 \cdot 10^5$  & \cellcolor{purpur}97,7  & 96,6 \\\hline
$-100$ до 100 & $10^6$  &  80,3 & \cellcolor{purpur}91,3 \\\hline
$-1000$ до 1000  & $10^6$  & 74,3 & \cellcolor{purpur}90 \\\hline
\end{tabular}
\end{table}

Из результатов эксперимента можно сделать вывод, что время и качество работы ЭА сильно зависит от выбора оператора мутации. При увеличении множества допустимых 
тестов, ЭА работает стабильнее и достигает лучших результатов, чем случайная генерация тестов.

\section{Выводы по главе \ref{chapter3}}
Были проведены эксперименты, демонстрирующие работу предложенного подхода для автоматизированного покрытия кода тестами на основе эволюционных алгоритмов. В 
результате экспериментов, проведенных на модельных задачах, было получено, что разработанный метод эффективен для покрытия даже сложных условий. Также 
применимость разработанного подхода была протестирована на олимпиадной задаче Huzita~Axiom~6. Было получено, что эффективность данного метода сильно зависит от 
выбора эволюционных операторов. Можно видеть, что при увеличении множества допустимых тестов результаты предложенного подхода превосходят результаты, 
полученные при покрытии кода случайными тестами.

% \startconclusionpage
Представленный в данной работе подход к представлению инвариантов —
по одному конструктору на случай инварианта —
приводит к неприятному разрастанию функций по обработке структуры данных.
Но данный подход позволил написать простые доказательства
с помощью интерактивного редактора, использующего систему типов
для указания типа требуемого терма.
Хотелось бы уметь обобщать такие представления инвариантов для
упрощения доказательств и уменьшения объема кода.


\printbibliography

% \startappendices
% \chapter{Обзор
% предметной области
}
\label{chapter1}

В данной главе производится обзор предметной области и даются определения используемых терминов.

\section{Функциональное программирование}

\emph{Функциональное программирование} — парадигма программирования,
являющаяся разновидностью декларативного программирования,
в которой программу представляют в виде функций
(в математическом смысле этого слова, а не в смысле, используемом в процедурном программировании),
а выполнением программы считают вычисление значений применения этих функций к заданным значениям.
Большинство функциональных языков программирования используют в своём основании лямбда-исчисление
(например, Haskell~\cite{HaskellLang}, Curry~\cite{CurryLang}, Agda~\cite{AgdaLang},
диалекты LISP~\cite{SchemeLang,ClojureLang,SICP}, SML~\cite{SMLLang}, OCaml\cite{OCamlLang}),
но существуют и функциональные языки явно не основанные на этом формализме
(например, препроцессор языка C и шаблоны в C++).

\section{Лямбда-исчисление}

\emph{Лямбда-исчисление} ($\lambda$-calculus)~— вычислительный формализм
с тремя синтаксическим конструкциями, называемыми \emph{пре-лямбда-термами}:
\begin{itemize}
\item \emph{вхождение переменной}: $v$. При этом $v \in V$, где $V$~— некоторое множество имён переменных;
\item \emph{лямбда-абстракция}: $\lambda x.A$, где $x$~— имя переменной, а $A$~— пре-лямбда-терм. При этом терм $A$ называют \emph{телом абстракции}, а $x$ перед точкой~— \emph{связыванием}.
\item \emph{лямбда-аппликация}: $B C$;
\end{itemize}
и одной операцией \emph{бета-редукции}.
При этом говорят, что вхождение переменной является \emph{свободным},
если оно не связано какой-либо абстракцией.
Множество пре-лямбда-термов обозначают $\Lambda^{-}$.
\emph{Лямбда-термы}~— это пре-лямбда-термы, факторизованные по отношению \emph{альфа-эквивалентности}.
Обозначение: $\Lambda = \Lambda^{-} / =_{\alpha} $.

\emph{Альфа-эквивалентность} ($\alpha$-equality) отождествляет два пре-лямбда-терма, если один из них может быть получен из другого путём некоторого \emph{корректного} переименовывания переменных~— переименования не нарушающего отношение связанности.

\emph{Бета-редукция} ($\beta$-reduction) для лямбда-терма $A$ выбирает в нём некоторую лямбда-аппликацию $B C$, содержащую лямбда-абстракцию в левой части $A$, и заменяет свободные вхождения переменной, связанной $A$, в теле самой $A$ на терм $C$.\footnote{В терминах пре-лямбда-термов это означает замену свободных вхождений в теле $A$ на пре-терм $C$ так, чтобы ни для каких переменных не нарушилось отношение связанности. То есть, в пре-терме $A$ следует корректно переименовать все связанные переменные, имена которых совпадают с именами свободных переменных в $C$.}

Два лямбда-терма $A$ и $B$ называются \emph{конвертабельными},
когда существует две последовательности бета-редукций, приводящих их к общему терму $C$.
Или, эквивалентно, когда термы $A$ и $B$ состоят с друг с другом в рефлексивно-симметрично-транзитивном замыкании отношения бета-редукции, также называемом отношением \emph{бета-эквивалентности}.

За более подробной информацией об этом формализме следует обращаться к~\cite{TTFP} и~\cite{Sorensen}.

\section{Лямбда-исчисление с простыми типами}
\begin{definition}
    Пусть $U$ — бесконечное счетное множество, элементы которого мы будем
    называть \emph{переменными типов}.
    Множество \emph{простых типов} $\Pi$ — множество, определенное грамматикой:

    $$ \Pi ::= U \mid (\Pi \to \Pi) $$

    Для обозначения элементов множества $\Pi$ используют буквы греческого алфавита:
    $\sigma, \tau \ldots $.
\end{definition}
\begin{definition}
    Множество контекстов $C$ — это множество всех множеств пар такого вида:
    $$ \{ x_1 : \tau_1 , \ldots , x_n : \tau_n \} $$
    где $ \tau_1 , \ldots , \tau_n \in \Pi$, а
    $ x_1 , \ldots , x_n \in V $ (переменные из $\Lambda$) и $x_i \neq x_j$ если $i \neq j$.
\end{definition}
\begin{definition}
    \emph{Домен} контекста $\Gamma$ = $ \{ x_1 : \tau_1 , \ldots , x_n : \tau_n \} $:
    $$ \operatorname{dom} (\Gamma) = \{ x_1 , \ldots , x_n \} $$ и $x_i \neq x_j$ при $i \neq j$.
\end{definition}
\begin{definition}
    Отношение \emph{типизации} (typability) $ \vdash $ на множестве
    $ C \times \Lambda \times \Pi $ определяется следующими правилами:

    $$ \frac {} {\Gamma, x : \tau \vdash x : \tau}
    \quad
    \frac{\Gamma, x : \sigma \vdash M : \tau}
    {\Gamma \vdash \lambda x . M : \sigma \to \tau}
    \quad
    \frac{\Gamma \vdash M : \sigma \to \tau \quad \Gamma \vdash N : \sigma}
    {\Gamma \vdash M N : \tau}
    $$
    В первом и втором правиле мы требуем $x \notin \operatorname{dom}(\Gamma).$
\end{definition}
\begin{definition}
    \emph{Лямбда-исчисление с простыми типами} или $\lambda^{\to}$ — это тройка
    $ (\Lambda , \Pi , \vdash) $.
    Чтобы отличать данное в этой работе определение системы $\lambda^{\to}$ от других вариантов,
    эту систему называют лямбда-исчисление с простыми типами \emph{по Карри}.
\end{definition}
За более подробной информацией об этом формализме следует
обращаться к~\cite{ChurchSTLC} и~\cite{Sorensen}.

\section{Алгебраические типы данных и сопоставление с образцом}

\emph{Алгебраический тип данных} — вид составного типа, то есть типа,
сформированного комбинированием других типов.
Комбинирование осуществляется с помощью алгебраических операций — сложения и умножения.

\emph{Сумма} типов $A$ и $B$ — дизъюнктное объединение исходных типов.
Значения типа-суммы обычно создаются с помощью \emph{конструкторов}.

\emph{Произведение} типов $A$ и $B$ — прямое произведение исходных типов,
кортеж типов.

\subsection{Рекурсивные типы данных}
\emph{Рекурсивный тип данных} — тип данных, в определении которого содержится
определяемый тип данных. Например, список элементов типа $A$:
$$ List\;A = Nil + (A \times List\;A) $$

В теории~\cite{TAPL} для введения рекурсивных типов используются $\mu$-типы.
\emph{Сырые} $\mu$-типы вводятся с помощью оператора $\mu$: $\mu X . T $.
При этом $T$ может содержать $X$.

\begin{definition}
Сырой $\mu$-тип $T$ называется \emph{сократимым} (contractive),
если для любого подвыражения $T$ вида $ \mu X. \mu X_1 \ldots \mu X_n . S $
тело $S$ не равняется $X$.

Сырой $\mu$-тип называется просто \emph{$\mu$-типом} ($\mu$ -type), если он сократим.
\end{definition}
Пример: список элементов типа $A$: $ List\;A = \mu X . Nil + (A \times X)$.

\subsection{Сопоставление с образцом}

\emph{Сопоставление с образцом} — способ обработки
объектов % | значений |
алгебраических типов данных, который идентифицирует значения по конструктору
и извлекает данные в соответствии с представленным образцом.

\section{Теория типов}

\emph{Теория типов} — раздел математики изучающий отношения типизации вида
$ M \colon \tau $ и их свойства. $M$ называется \emph{термом} или \emph{выражением},
а $\tau$ — типом терма $M$.

Теория типов также изучает правила для \emph{переписывания} термов — замены
подтермов в выражениях другими термами.
Такие правила также называют правилами \emph{редукции} или \emph{конверсии} термов.
Редукцию терма $x$ в терм $y$ записывают: $x \to y$.
Также рассматривают транзитивное замыкание отношения редукции: $ \xrightarrow{*} $.
Например, термы $2 + 1$ и $3$ — разные термы, но первый редуцируется во второй:
$2 + 1 \xrightarrow{*} 3$.
Если для терма $x$ не существует терма $y$, для которого $x \to y$,
то говорят, что терм $x$ — в \emph{нормальной форме}.

\subsection{Отношение конвертабельности}

Два терма $x$ и $y$ называются \emph{конвертабельными},
если существует терм $z$ такой, что $x \xrightarrow{*} z$ и $y \xrightarrow{*} z$. Обозначают  $x \xleftrightarrow{*} y$.
Например, $1+2$ и $2+1$ — конвертабельны, как и термы
$x + (1 + 1)$ и $x + 2$. Однако, $x+1$ и $1+x$ (где $x$ — свободная переменная)
— не конвертабельны, так как оба представлены в нормальной форме.
Конвертабельность — рефлексивно-транзитивно-симметричное замыкание отношения
редукции.

\subsection{Интуиционистская теория типов}

Интуиционистская теория типов (теория типов Мартина-Лёфа)
основана на математическом конструктивизме~\cite{MLTT}.

Операторы для типов в ИТТ:
\begin{itemize}
    \item $\Pi$-тип (пи-тип) — зависимое произведение, обобщение типов функций ($ X \to Y $),
        в которых тип результата зависит от значения аргумента: $\Pi_{x : X} Y(x)$.
        Например, если $\operatorname{Vec}(A, n)$ — тип кортежей из $n$ элементов типа $A$,
        $\mathbb N$ — тип натуральных чисел, то
        $\Pi_{n \mathbin{:} {\mathbb N}} \operatorname{Vec}(A, n)$ 
        — тип функции, которая по натуральному числу $n$ возвращает кортеж из
        $n$ элементов типа $A$.
    \item $\Sigma$-тип — зависимая пара $\Sigma_{x : A} B(x)$.
        Второй элемент в зависимой паре зависит от первого.
        Например, тип $\Sigma_{n \mathbin{:} {\mathbb N}} \operatorname{Vec}(A, n)$ — тип 
        пары из числа $n$ и кортежа из $n$ элементов типа $A$.
    \item Пусть $A$ — множество конструкторов, $B$ — селектор на $A$.
        Элементы множества $A$ представляют разные способы сформировать
элемент в $W_{a : A} B(a) $, а $B(a)$ представляют части дерева, сформированные с помощью $a$.
        $W_{a : A} B(a) $  — рекурсивный тип, построенный с помощью конструкторов $B(a)$,
        который можно представить в виде \emph{фундированных деревьев} (well-founded trees)~\cite{WTypes}.
\end{itemize}
Базовые типы в ИТТ:
$\bot$ или $0$ — пустой тип, не содержащий ни одного элемента;
$\top$ или $1$ — единичный тип, содержащий единственный элемент.

\section{Унификация}

\emph{Унификатор} для термов $A$ и $B$ — подстановка $S$, действующая на их
свободные переменные, такая что $S(A) \equiv S(B)$.

\emph{Унификация} — процесс поиска унификатора.

\section{Agda}
\emph{Agda}~\cite{AgdaLang}~—  чистый функциональный язык программирования с зависимыми типами.
В Agda есть поддержка модулей:
\begin{code}\>\<%
\\
\>\AgdaKeyword{module} \AgdaModule{AgdaDescription} \AgdaKeyword{where}\<%
\\
\>\<\end{code} В коде на Agda широко используются символы Unicode.
Тип натуральных чисел — \D{ℕ}.
\begin{code}\>\<%
\\
\>\AgdaKeyword{data} \AgdaDatatype{ℕ} \AgdaSymbol{:} \AgdaPrimitiveType{Set} \AgdaKeyword{where}\<%
\\
\>[0]\AgdaIndent{2}{}\<[2]%
\>[2]\AgdaInductiveConstructor{zero} \AgdaSymbol{:} \AgdaDatatype{ℕ}\<%
\\
\>[0]\AgdaIndent{2}{}\<[2]%
\>[2]\AgdaInductiveConstructor{succ} \AgdaSymbol{:} \AgdaDatatype{ℕ} \AgdaSymbol{→} \AgdaDatatype{ℕ}\<%
\\
\>\<\end{code} \AgdaHide{
\begin{code}\>\<%
\\
\>\AgdaSymbol{\{-\#} \AgdaKeyword{BUILTIN} NATURAL \AgdaDatatype{ℕ} \AgdaSymbol{\#-\}}\<%
\\
\>\AgdaSymbol{\{-\#} \AgdaKeyword{BUILTIN} ZERO \AgdaInductiveConstructor{zero} \AgdaSymbol{\#-\}}\<%
\\
\>\AgdaSymbol{\{-\#} \AgdaKeyword{BUILTIN} SUC \AgdaInductiveConstructor{succ} \AgdaSymbol{\#-\}}\<%
\\
\>\<\end{code}
}
В Agda функции можно определять как mixfix операторы.
Пример — сложение натуральных чисел:
\begin{code}\>\<%
\\
\>\AgdaFunction{\_+\_} \AgdaSymbol{:} \AgdaDatatype{ℕ} \AgdaSymbol{→} \AgdaDatatype{ℕ} \AgdaSymbol{→} \AgdaDatatype{ℕ}\<%
\\
\>\AgdaInductiveConstructor{zero} \AgdaFunction{+} \AgdaBound{b} \AgdaSymbol{=} \AgdaBound{b}\<%
\\
\>\AgdaInductiveConstructor{succ} \AgdaBound{a} \AgdaFunction{+} \AgdaBound{b} \AgdaSymbol{=} \AgdaInductiveConstructor{succ} \AgdaSymbol{(}\AgdaBound{a} \AgdaFunction{+} \AgdaBound{b}\AgdaSymbol{)}\<%
\\
\>\<\end{code}
Символы подчеркивания обозначают места для аргументов.
% Система типов \textit{Agda} позволяет ... 
% В отличие от \textit{Haskell}, в \textit{Agda} имеется ... 

Зависимые типы позволяют определять типы, зависящие (индексированные) от значений
других типов. Пример — список, индексированный своей длиной:
\begin{code}\>\<%
\\
\>\AgdaKeyword{data} \AgdaDatatype{Vec} \AgdaSymbol{(}\AgdaBound{A} \AgdaSymbol{:} \AgdaPrimitiveType{Set}\AgdaSymbol{)} \AgdaSymbol{:} \AgdaDatatype{ℕ} \AgdaSymbol{→} \AgdaPrimitiveType{Set} \AgdaKeyword{where}\<%
\\
\>[0]\AgdaIndent{2}{}\<[2]%
\>[2]\AgdaInductiveConstructor{nil} \<[7]%
\>[7]\AgdaSymbol{:} \AgdaDatatype{Vec} \AgdaBound{A} \AgdaInductiveConstructor{zero}\<%
\\
\>[0]\AgdaIndent{2}{}\<[2]%
\>[2]\AgdaInductiveConstructor{cons} \AgdaSymbol{:} \AgdaSymbol{∀} \AgdaSymbol{\{}\AgdaBound{n}\AgdaSymbol{\}} \AgdaSymbol{→} \AgdaBound{A} \AgdaSymbol{→} \AgdaDatatype{Vec} \AgdaBound{A} \AgdaBound{n} \AgdaSymbol{→} \AgdaDatatype{Vec} \AgdaBound{A} \AgdaSymbol{(}\AgdaInductiveConstructor{succ} \AgdaBound{n}\AgdaSymbol{)}\<%
\\
\>\<\end{code}
В фигурные скобки заключаются неявные аргументы.

Такое определение позволяет нам описать функцию $ \F{head} $ для такого списка, которая не может бросить исключение:
\begin{code}\>\<%
\\
\>\AgdaFunction{head} \AgdaSymbol{:} \AgdaSymbol{∀} \AgdaSymbol{\{}\AgdaBound{A}\AgdaSymbol{\}} \AgdaSymbol{\{}\AgdaBound{n}\AgdaSymbol{\}} \AgdaSymbol{→} \AgdaDatatype{Vec} \AgdaBound{A} \AgdaSymbol{(}\AgdaInductiveConstructor{succ} \AgdaBound{n}\AgdaSymbol{)} \AgdaSymbol{→} \AgdaBound{A}\<%
\\
\>\<\end{code}
У аргумента функции $ \F{head} $ тип $ \D{Vec}\,A\,(\DC{succ}\,n) $, то есть вектор, в котором есть хотя бы один элемент.
Это позволяет произвести сопоставление с образцом только по конструктору $ \DC{cons} $:
\begin{code}\>\<%
\\
\>\AgdaFunction{head} \AgdaSymbol{(}\AgdaInductiveConstructor{cons} \AgdaBound{a} \AgdaBound{as}\AgdaSymbol{)} \AgdaSymbol{=} \AgdaBound{a}\<%
\\
\>\<\end{code}



\section{Индуктивные семейства}

\begin{definition}
\emph{Индуктивное семейство}~\cite{DybjerIndFam, RefiningIT}~— это индуктивный тип данных,
который может зависеть от других типов и значений.
Тип или значение, от которого зависит зависимый тип, называют \emph{индексом}.
\end{definition}

Одной из областей применения индуктивных семейств являются системы интерактивного
доказательства теорем.

Индуктивные семейства позволяют формализовать математические структуры,
кодируя утверждения о структурах в них самих,
тем самым перенося сложность из доказательств в определения.

\section{Использование индуктивных семейств в структурах данных}
В работах~\cite{HongweiXi, McBridePivotal} приведены различные подходы
в использовании индуктивных семейств в реализации структур данных
и доказательств их свойств.

Пример задания структуры данных и инвариантов — тип данных AVL-дерева
и тип данных для хранения баланса высоты поддеревьев в AVL-дереве~\cite{AVLTree}.

\AgdaHide{
\begin{code}\>\<%
\\
\>\AgdaKeyword{module} \AgdaModule{AVLBalance} \AgdaKeyword{where}\<%
\\
\>[0]\AgdaIndent{2}{}\<[2]%
\>[2]\AgdaKeyword{open} \AgdaKeyword{import} \AgdaModule{AgdaDescription}\<%
\\
\>[0]\AgdaIndent{2}{}\<[2]%
\>[2]\AgdaKeyword{infix} \AgdaNumber{4} \_∼\_\<%
\\
\>[0]\AgdaIndent{2}{}\<[2]%
\>[2]\AgdaFunction{testˡ} \AgdaSymbol{:} \AgdaDatatype{ℕ}\<%
\\
\>[0]\AgdaIndent{2}{}\<[2]%
\>[2]\AgdaFunction{testˡ} \AgdaSymbol{=} \AgdaInductiveConstructor{succ} \AgdaInductiveConstructor{zero}\<%
\\
\>\<\end{code}
}
Если $m \sim n$, то разница между $m$ и $n$ не больше чем один:
\begin{code}\>\<%
\\
\>[0]\AgdaIndent{2}{}\<[2]%
\>[2]\AgdaKeyword{data} \AgdaDatatype{\_∼\_} \AgdaSymbol{:} \AgdaDatatype{ℕ} \AgdaSymbol{→} \AgdaDatatype{ℕ} \AgdaSymbol{→} \AgdaPrimitiveType{Set} \AgdaKeyword{where}\<%
\\
\>[2]\AgdaIndent{4}{}\<[4]%
\>[4]\AgdaInductiveConstructor{∼+} \AgdaSymbol{:} \AgdaSymbol{∀} \AgdaSymbol{\{}\AgdaBound{n}\AgdaSymbol{\}} \AgdaSymbol{→} \<[21]%
\>[21]\AgdaBound{n} \AgdaDatatype{∼} \AgdaNumber{1} \AgdaFunction{+} \AgdaBound{n}\<%
\\
\>[2]\AgdaIndent{4}{}\<[4]%
\>[4]\AgdaInductiveConstructor{∼0} \AgdaSymbol{:} \AgdaSymbol{∀} \AgdaSymbol{\{}\AgdaBound{n}\AgdaSymbol{\}} \AgdaSymbol{→} \<[21]%
\>[21]\AgdaBound{n} \AgdaDatatype{∼} \AgdaBound{n}\<%
\\
\>[2]\AgdaIndent{4}{}\<[4]%
\>[4]\AgdaInductiveConstructor{∼-} \AgdaSymbol{:} \AgdaSymbol{∀} \AgdaSymbol{\{}\AgdaBound{n}\AgdaSymbol{\}} \AgdaSymbol{→} \AgdaNumber{1} \AgdaFunction{+} \AgdaBound{n} \AgdaDatatype{∼} \AgdaBound{n}\<%
\\
\>\<\end{code}


В работе~\cite{McBridePivotal} представлен способ обобщения
упорядоченных структур данных
(таких как отсортированные списки и деревья поиска)
и использование этого метода для реализации 2-3 деревьев.

\section{Выводы по главе~\ref{chapter1}}

В этой главе были рассмотрены некоторые существующие подходы к построению структур данных
с использованием индуктивных семейств.
Кратко описаны особенности языка программирования \textit{Agda},
который использован в этой работе.

% \begin{appendices}
%     \chapter{Some Appendix}
%     The contents...
% \end{appendices}

\end{document}
