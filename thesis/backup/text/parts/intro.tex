\startprefacepage

При разработке современного программного обеспечения значительную часть времени и усилий занимает процесс тестирования. Зачастую он сводится к анализу поведения 
программы на специально подобранном наборе тестов. Поскольку перебор всех возможных ситуаций неосуществим на практике, стараются выбрать небольшое число 
тестов, таких чтобы по поведению программы на них можно было судить о ее поведении в целом.

Используются различные метрики покрытия кода для количественной оценки качества набора тестов. Качественным считается набор тестов, удовлетворяющий некоторому 
критерию полноты. Зачастую критерий полноты определяется с помощью выбранной метрики покрытия кода.

Автоматизация процесса построения набора тестов позволит существенно сократить затраты на тестирование. Существует множество подходов к автоматизированному 
построению набора тестов. Однако лишь немногие из них имеют программную реализацию, например $\text{Microsoft Pex}$. Все эти реализации нацелены на модульное 
тестирование, при котором покрытие фрагментов программы происходит независимо друг от друга. Вследствие этого большое число генерируемых тестов содержат 
недопустимые значения для тестируемых фрагментов кода, так как не учитывается внутренняя структура программы. 

При тестировании некоторых программ, таких как решения олимпиадных задач, нет необходимости проводить модульное тестирование. Для полноценного покрытия тестами 
олимпиадных задач необходимо генерировать тесты, подходящие под условие задачи. В данной работе разрабатывается метод построения набора тестов на основе 
эволюционных алгоритмов, учитывающий особенности всей программы в целом. 