\chapter{Результаты} 
\label{chapter3}

В данном разделе описываются результаты экспериментов, демонстрирующих работу предложенного метода генерации покрывающего набора тестов. 

\section{Покрытие тестами модельных задач}
Исследования проводились на пяти модельных задачах, взятых с сайта инструментального средства Microsoft Pex \texttt{http://pexforfun.com}. 

\subsection{Описание модельных задач}
Для каждой задачи приводится исходный код на языке \textit{C\#}, описывается способ кодирования теста и оператор мутации. Для построения покрывающего набора 
тестов использовалась \texttt{(1+1)}-эволюционная стратегия.

В рассматриваемых задачах при применении оператора мутации для изменения целого числа \texttt{x} к нему добавлялось число вида $(r(19) - 9) \cdot 10 ^{r(10)}$, 
где $r(a)$~--- случайное целое число в диапазоне $[0, a)$. Если \texttt{x} должен находиться в диапазоне от $-10^5$ до $10^5$, то он заменяется на $\max(-10^5, 
\min(10^5, x + (r(19) - 9) \cdot 10^{r(4)}))$. 

\subsubsection{Задача 1}
\label{pex01}
Код тестируемой программы приведен на листинге~\ref{lst:pex-1-source}. Наиболее сложным для покрытия является случай, когда сумма элементов массива равняется 
42. В качестве тестов генерируются целочисленные массивы длины шесть. Оператор мутации выбирает случайный элемент массива и изменяет его описанным ранее 
способом.

\begin{snippet}[language=C++,caption={Код задачи 1 с сайта pexforfun},label={lst:pex-1-source}]
using System;

public class Program {
  //#What values of v to cause the exception? Ask Pex to find out!#
  public static int Puzzle(int[] v) {
    int sum = 0;
    foreach (int x in v)
      sum += x;
    if (sum == 42)
      throw new Exception("hidden bug!");
    return sum;
  }
}
\end{snippet}

\subsubsection{Задача 2}

Код тестируемой программы приведен на листинге~\ref{lst:pex-2-source}. Сложность данной задачи заключается в покрытии случая, когда элемент списка, следующий 
за указанным, на единицу больше. Тестом является список, состоящий из шести целых чисел, и число из множества $\{0, 1, 2, 3, 4\}$~--- в качестве индекса 
элемента в списке. Оператор мутации случайным образом выбирает другой индекс, либо изменяет элемент списка, выбранный случайным образом.  

\begin{snippet}[language=C++,caption={Код задачи 2 с сайта pexforfun},label={lst:pex-2-source}]
using System;
using System.Collections.Generic;

public class Program
{
  //# What values of list and i can cause exceptions? Ask Pex to find out!#
  public static void Puzzle(List<int> list, int i)
  {
    if (list[i] + 1 == list[i + 1])
      throw new Exception("hidden bug!"); 
  }
}
\end{snippet}

\subsubsection{Задача 3}
                   
Код тестируемой программы приведен на листинге~\ref{lst:pex-3-source}. Сложнее всего покрыть случай, когда сумма выбранного элемента массива и 27277 равняется 
42. Тест представляется в виде массива из шести целых чисел и числа из множества $\{0, 1, 2, 3, 4, 5\}$~--- индекса элемента в массиве. Оператор мутации 
случайным образом выбирает другой индекс, либо изменяет элемент массива, выбранный случайным образом.  

\begin{snippet}[language=C++,caption={Код задачи 3 с сайта pexforfun},label={lst:pex-3-source}]
using System;

public class Program {
  //# What values of v and i can cause an exception? Ask Pex to find out!#
  public static void Puzzle(int[] v, int i) {
    if (v[i] + 27277 == 42)
      throw new Exception("hidden bug!"); 
  }
}
\end{snippet}

\subsubsection{Задача 4}
                   
Код тестируемой программы приведен на листинге~\ref{lst:pex-4-source}. Чтобы полностью покрыть данную задачу, необходимо подобрать корень линейного уравнения. 
В качестве теста берется целое число в диапазоне от $-10^5$ до $10^5$. Оператор мутации изменяет текущее решение описанным ранее образом.

\begin{snippet}[language=C++,caption={Код задачи 4 с сайта pexforfun},label={lst:pex-4-source}]
using System;

public class Program 
{
  public static void Puzzle(int x)
  {
    // What value of x solves this equation? Ask Pex to find out!
    if (x * 3 + 27 == 153)
      Console.WriteLine("equation solved");
  }
}
\end{snippet}

\subsubsection{Задача 5}
                                                            
Код тестируемой программы приведен на листинге~\ref{lst:pex-5-source}. Сложнее всего покрыть тестом строку, в которой производится вывод на консоль.
Это эквивалентно решению нелинейного уравнения второй степени в целых числах с двумя переменными и наличием ограничений. Тест представляется в виде кортежа из 
двух целых чисел в диапазоне от $-10^5$ до $10^5$. Оператор мутации изменяет одно из них.

\begin{snippet}[language=C++,caption={Код задачи 5 с сайта pexforfun},label={lst:pex-5-source}]
using System;

public class Program {
  public static void Puzzle(int x, int y) {
    //# What values of x and y solve this equation? Ask Pex to find out!#
    if (x >= 0 && x <= 100 &&
        y >= 0 && y <= 100 &&
        x * y - 37 * y + 71 * x - 2627 == 0)        
      Console.WriteLine("equation solved"); 
  }
}
\end{snippet}

\subsection{Результаты эксперимента}

Для каждой из рассмотренных задач было проведено 1000 запусков эволюционного алгоритма. В результате каждого из запусков был сгенерирован покрывающий набор 
тестов. В таблице~\ref{pexres} приведены минимальное, среднее и максимальное число вычислений функции приспособленности (ФП).

\begin{table}[h!]
\caption{Число вычислений ФП на модельных задачах} \label{pexres}
\begin{tabular}{|p{5em}|c|c|c|}
\hline
№ задачи & ФП, мин. & ФП, среднее & ФП, макс. \\\hline
%Pex #01
1  & 33  &  340 & 1449 \\\hline
%Pex #02
2 & 2  &  660& 3909 \\\hline
%Pex #03
3  & 286  &  2604 & 13078 \\\hline
%Pex #04
4  & 20  &  201 & 679 \\\hline
%Pex #05
5  & 298  &  1429 & 4086 \\\hline
\end{tabular}
\end{table}

Покрытие модельных задач было осуществлено с помощью тестов, сгенерированных случайным образом. При этом, чтобы было возможным полное покрытие заданного кода, 
пространство поиска тестов было уменьшено. Так, например, для задачи \ref{pex01} массив заполнялся целыми числами в диапазоне от $-5 \cdot 10^5$ до $5 \cdot 
10^5$. При этом на каждом запуске генерировалось в $10$ раз больше тестов, чем максимальное число вызовов ФП для соответствующей задачи. В таблице 
\ref{randompex} приведены полученные результаты, усредненные по 1000 запускам.

\begin{table}
\caption{Результаты покрытия модельных задач случайными тестами} \label{randompex}
\begin{tabular}{|p{5em}|c|c|}
\hline
№ задачи & Покрытие, \% & Число тестов на каждом запуске\\\hline
% Pex #01
1 & 75.3 & 15000\\\hline
% Pex #02
2 & 52.7 & 40000\\\hline
% Pex #03
3 & 55.6 & 130000\\\hline
% Pex #04
4 & 51.4 & 7000\\\hline
% Pex #05
5 & 66.7 & 41000\\\hline
\end{tabular}
\end{table}

Из результатов эксперимента можно сделать вывод, что применение эволюционного алгоритма для генерации тестов, покрывающих заданные строки кода, весьма 
эффективно даже для сложных условий.

\section{Покрытие тестами олимпиадных задач}
Исследования проводились на основе задачи  Huzita Axiom 6, предложенной на NEERC 2011. 

\subsection{Условие задачи}
Заданы две прямые $l_1$ и $l_2$ и две точки $p_1$ и $p_2$. Необходимо найти прямую, по которой можно сложить плоскость так, что точка $p_1$ попадет на прямую 
$l_1$, а точка $p_2$ попадет на прямую $l_2$. Прямые задаются с помощью двух точек. Пример условия задачи приведен на рис.~\ref{huzita_pic}.
\begin{figure}[h!]
 \center{\includegraphics[width=0.8\textwidth]{huzita.pdf}}
 \caption{Иллюстрация к задаче  Huzita Axiom 6}
 \label{huzita_pic}
\end{figure}

\subsection{Описание эксперимента}

Для тестирования решений задачи был разработан интерфейс \textit{Solution}, приведенный на листинге~\ref{lst:huzita_solution}. Решения реализуют данный 
интерфейс. Входные данные(тест) передаются в виде строки, а результат выдается в виде списка строк.

\begin{snippet}[language=Java,caption={Интерфейс решения задачи Huzita Axiom 6},label={lst:huzita_solution}]
public interface Solution {
    public List<String> solve(String input) throws IOException;
}
\end{snippet}

Для построения покрывающего набора тестов использовалась \texttt{(1+1)}-эволюционная стратегия. Пример конфигурации ЭА, приведен на 
листинге~\ref{lst:huzita_config}.

\begin{snippet}[language=Scala,caption={Конфигурация эволюционного алгоритма для задачи  Huzita Axiom 6},label={lst:huzita_config}]
class HuzitaConfig[C <: Solution](val cl : Class[C], override val maxGenerationCount : Long) 
  extends  TargetAwareComputationStateImpl
  with CoverageConfig[TestData, DistDiffFitness]
  with OnePlusOneES
  with TraceProvider
  with Mutation
  with FastRandomDelegate
  with CoverageFitnessComparator.LeastDistLeastAbsDiff[DistDiffFitness]
  with GenotypeGenerator
  with GenotypeWrapper
  with TotalFitnessCallCount
{
  private[this] def nestedClasses(top : Class[_]) : Seq[Class[_]] = {
    top.getDeclaredClasses.toSeq.map(x => nestedClasses(x)).flatten :+ top
  }

  override def targetMethods: Seq[Method] = nestedClasses(cl).map(_.getDeclaredMethods).flatten

  override def targetConstructors: Seq[Constructor[_]] = nestedClasses(cl).map(_.getDeclaredConstructors).flatten

  def newGenotype(): TestData = TestData.genTestData()

  private[this] lazy val mh =  MethodHandles.lookup().findVirtual(cl, "solve", MethodType.methodType(classOf[java.util.List[String]], classOf[String]))

  def wrap(genotype: TestData): Seq[AnyRef] = genotype.toTestString :: Nil

  def mutate(source: TestData): TestData = source.mutate()

  def evaluateFitness(genotype: TestData): DistDiffFitness = computeFitness(wrap(genotype))

  protected def invoke(args: Seq[AnyRef]) {
    mh.invokeWithArguments(cl.newInstance() +: args : _*)
  } 
} 
\end{snippet}

Точка задается двумя целочисленными координатами. При мутации точки изменяются обе ее координаты. Линия определяется с помощью двух точек, и при мутации 
изменяется каждая из них. Тест, используемый в качестве особи ЭА, задается с помощью двух прямых~--- $l_1$ и $l_2$~--- и двух точек~---$p_1$ и $p_2$. 
Рассматривается пять типов операторов мутации тестовых данных.
Мутируют:
\begin{enumerate}
 \item либо $l_1$ и $p_1$, либо $l_2$ и $p_2$;
 \item либо $l_1$ и $l_2$, либо $p_1$ и $p_2$;
 \item одна из $l_1$, $l_2$, $p_1$, $p_2$;
 \item все прямые и все точки;
 \item одна из $l_1$ и $p_1$ и одна из $l_2$ и $p_2$.
\end{enumerate}

\subsection{Результаты}

Для каждого эксперимента было проведено по 1000 запусков. В таблицах~\ref{huzita_10_res}, \ref{huzita_100_res} и \ref{huzita_1000_res} приведены результаты 
работы ЭА, когда координаты точек лежат в диапазоне от $-10$ до 10, от $-100$ до 100 и от $-1000$ до 1000 соответственно. В первой колонке указан номер 
оператора мутации, во второй~--- среднее число вычислений ФП, а в третьей~--- усредненный процент покрытия кода. Можно видеть, что лучшие результаты были 
получены при использовании пятого оператора мутации.

\begin{table}[h!]
\caption{ЭА, диапазон от $-10$ до 10} \label{huzita_10_res}
\begin{tabular}{|p{5em}|c|c|}
\hline
№ мутации &  ФП, среднее & Покрытие, \% \\\hline
1  & $2,5 \cdot 10^4$ &  95,8 \\\hline
2  & $2,9 \cdot 10^4$  &  95  \\\hline
3  & $4,5 \cdot 10^4$  &  89,3 \\\hline
4  & $2,5 \cdot 10^4$  &  95,9 \\\hline
5  & $2,2 \cdot 10^4$  &   \cellcolor{purpur}96,6 \\\hline
\end{tabular}
\end{table}

\begin{table}[h!]
\caption{ЭА, диапазон от $-100$ до 100} \label{huzita_100_res}
\begin{tabular}{|p{5em}|c|c|}
\hline
№ мутации &  ФП, среднее & Покрытие, \% \\\hline
1  & $5,7 \cdot 10^5$ & 87,3 \\\hline
2  & $4,7 \cdot 10^5$  &  89,7  \\\hline
3  & $5,7 \cdot 10^5$  &  87,3 \\\hline
4  & $6,7 \cdot 10^5$  &  88,3 \\\hline
5  & $4,2 \cdot 10^5$  &  \cellcolor{purpur}91,3 \\\hline
\end{tabular}
\end{table}

\begin{table}[h!]
\caption{ЭА, диапазон от $-1000$ до 1000} \label{huzita_1000_res}
\begin{tabular}{|p{5em}|c|c|}
\hline
№ мутации &  ФП, среднее & Покрытие, \% \\\hline
1  & $7,4 \cdot 10^5$  & 81,7 \\\hline
2  & $6,7 \cdot 10^5$  &  87,3  \\\hline
3  & $6,2 \cdot 10^5$ & 88 \\\hline
4  & $7,9 \cdot 10^5$  &  79,3 \\\hline
5  & $5,9 \cdot 10^5$ &  \cellcolor{purpur}90 \\\hline
\end{tabular}
\end{table}

В таблице~\ref{huzita_random_res} приведены результаты покрытия кода случайными тестами в сравнении с результатами ЭА. В первой колонке указан диапазон 
координат, во второй~--- число случайно генерируемых тестов, в третьей~--- усредненный процент покрытия кода случайными тестами, а в четвертой~--- лучший 
усредненный процент покрытия кода с помощью ЭА.

\begin{table}
\caption{Покрытие решения Huzita Axiom 6 случайными тестами} \label{huzita_random_res}
\begin{tabular}{|p{7em}|c|c|c|}
\hline
Диапазон &  Число тестов & Покрытие, \% & Покрытие ЭА, \%\\\hline
$-10$ до 10  & $5 \cdot 10^5$  & \cellcolor{purpur}97,7  & 96,6 \\\hline
$-100$ до 100 & $10^6$  &  80,3 & \cellcolor{purpur}91,3 \\\hline
$-1000$ до 1000  & $10^6$  & 74,3 & \cellcolor{purpur}90 \\\hline
\end{tabular}
\end{table}

Из результатов эксперимента можно сделать вывод, что время и качество работы ЭА сильно зависит от выбора оператора мутации. При увеличении множества допустимых 
тестов, ЭА работает стабильнее и достигает лучших результатов, чем случайная генерация тестов.

\section{Выводы по главе \ref{chapter3}}
Были проведены эксперименты, демонстрирующие работу предложенного подхода для автоматизированного покрытия кода тестами на основе эволюционных алгоритмов. В 
результате экспериментов, проведенных на модельных задачах, было получено, что разработанный метод эффективен для покрытия даже сложных условий. Также 
применимость разработанного подхода была протестирована на олимпиадной задаче Huzita~Axiom~6. Было получено, что эффективность данного метода сильно зависит от 
выбора эволюционных операторов. Можно видеть, что при увеличении множества допустимых тестов результаты предложенного подхода превосходят результаты, 
полученные при покрытии кода случайными тестами.
