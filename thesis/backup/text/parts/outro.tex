\startconclusionpage

В данной работе создана платформа для автоматизированного покрытия кода тестами на основе эволюционных алгоритмов. Разработан новый способ построения функции 
приспособленности: используется не опорная траектория, а рассматривается приближение к цели по многим направлениям.

Предложенный метод был успешно применен для покрытия тестами ряда модельных задач. Было проведено сравнение разработанного подхода с методом генерации 
случайных тестов. В результате экспериментов было получено, что в отличие от метода генерации случайных тестов, предложенный подход обеспечивает полное 
покрытие кода.

Кроме того, разработанный метод был применен для покрытия тестами решения олимпиадной задачи Huzita~Axiom~6. В ходе экспериментов было установлено, что 
результат работы предложенного подхода зависит от выбора эволюционных операторов. Предложенный метод показал хорошие результаты даже для больших диапазонов 
допустимых тестов, в то время как метод генерации случайных тестов применим лишь для малых диапазонов.

Таким образом, данная работа полностью удовлетворяет поставленным требованиям.