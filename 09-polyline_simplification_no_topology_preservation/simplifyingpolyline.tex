\begin{problem}{Упрощение полигональной цепи}{standard input}{standard output}{2 секунды}{256 мегабайт}

Полигональная цепь — упорядоченное множество отрезков, начало каждого следующего совпадает с предыдущим.
Ваша задача — упростить полигональную цепь с помощью алгоритма Дугласа-Пеккера (\emph{Douglas-Peucker}) так,
чтобы расстояние от исходной цепи до получившейся было не больше $\varepsilon$.


\InputFile

На первой строчке входного файла записано число $\varepsilon$.
Во второй строчке $n$ — количество отрезков в множестве.
На последующих $n + 1$ строчках записаны пары вещественных чисел $(x_i, y_i)$, задающие координаты концов отрезков.

\OutputFile

На первой строке выходного файла выведите $m$ — количество точек в упрощенной полигональной цепи.
На второй строке через пробел выведите $m$ чисел — номера точек, являющихся вершинами выпуклой оболочки.
Нумерация точек начинается с единицы и задаётся в порядке следования их координат во входном файле.

\Examples

\begin{example}%
\exmp{
5
0.0 0.0
0.0 1.0
1.0 0.0
1.0 1.0
0.5 0.5
}{
4
1 2 4 3
}%
\end{example}

\end{problem}